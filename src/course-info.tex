\documentclass[11pt]{article}

\usepackage{fullpage}
\usepackage{microtype}
\usepackage{newtxtext}
\usepackage[T1]{fontenc}
\usepackage[pagebackref=true]{hyperref}

\usepackage{mathtools,amsfonts,amssymb}
\usepackage[amsmath,amsthm,thmmarks,hyperref]{ntheorem}
\usepackage[capitalize,nameinlink,noabbrev]{cleveref}
\usepackage{import}

\usepackage{fancyhdr}
\usepackage{enumitem}

\import{common/}{head.tex}

\chead{\large \textbf{Course Information\\and Syllabus}}

\lhead{\small
  \textbf{\href{https://github.com/cpeikert/LatticesInCryptography}%
    {Lattices in Cryptography}\\University of Michigan, Fall 2022}}

\rhead{\small \textbf{Instructor:
    \href{http://www.eecs.umich.edu/~cpeikert/}{Chris Peikert}}}

\setlength{\headheight}{27pt}
\setlength{\headsep}{20pt}

\pagestyle{plain}               % default: no special header

\begin{document}

\thispagestyle{fancy}           % first page should have special header

\section{General Information}
\label{sec:general-information}

\emph{Point lattices} in $\R^{n}$ have proven remarkably useful in
cryptography, both for cryptanalysis (breaking codes) and more
recently for constructing cryptosystems with unique security and
functionality properties.

This graduate-level seminar will cover classical results, exciting
recent developments, and important open problems.  Specific topics,
depending on time and level of interest, will be drawn from:
\begin{itemize}[itemsep=0pt]
\item Mathematical background and basic results
\item The LLL algorithm, Coppersmith's method, and applications to
  cryptanalysis
\item Complexity of lattice problems: NP-hardness, algorithms and
  other upper bounds
\item Gaussians, harmonic analysis, and the smoothing parameter
\item Worst-case-to-average-case reductions and the SIS/LWE problems
\item Basic cryptographic constructions: one-way functions, encryption
  schemes, digital signatures
\item ``Exotic'' cryptographic constructions: ID-based encryption,
  fully homomorphic encryption, and more
\item ``Algebraic'' (ring-based) cryptographic reductions and
  primitives
\end{itemize}

\subsection{Materials}
\label{sec:materials}

The public course web page with lecture notes, homeworks, and other
materials is at {\small
  \url{https://github.com/cpeikert/LatticesInCryptography}}. For
assignment submission, grading, discussions, etc., we will use the
course Canvas site at
\url{https://umich.instructure.com/courses/559604}.

There is no required textbook for this class; lectures, notes, and
research papers are the main sources of content. Students may also
wish to refer to the following excellent sources:
\begin{itemize}
\item Oded Regev's course
  \href{http://www.cims.nyu.edu/~regev/teaching/lattices_fall_2009/index.html}{\emph{Lattices
      in Computer Science}}
\item Micciancio and Goldwasser's book
  \href{http://link.springer.com/book/10.1007/978-1-4615-0897-7/page/1}{\emph{Complexity
      of Lattice Problems: A Cryptographic Perspective}}
\end{itemize}

Instructor office hours will be held on \textbf{Tuesdays at 11am}
(with some exceptions, to be announced ahead of time), or by
appointment.

\subsection{Prerequisites}
\label{sec:prerequisites}

There are no formal prerequisite classes.  However, this course is
mathematically rigorous and fast-paced, hence the main requirement is
\emph{mathematical maturity}.  Specifically, students should be
comfortable with devising and writing correct formal proofs (and
finding the flaws in incorrect ones!), devising and analyzing
algorithms, and working with probability.

A previous course in cryptography (e.g., Applied or Theoretical
Cryptography) is very helpful but is not required.  No previous
familiarity with lattices will be assumed.  \emph{Highly recommended}
courses---the more the better---include: EECS~477 or~586 (Algorithms),
EECS~574 (Computational Complexity Theory), EECS 475/575 (Introduced
to/Advanced Cryptography).  The instructor reserves the right to limit
enrollment to students who have the necessary background.

\section{Course Policies}
\label{sec:policies}

\subsection{Grading}
\label{sec:grading}

Grades will be determined roughly as follows:
\begin{itemize}
\item[(50\%)] Homework assignments (about 4), due approximately every
  two weeks.  Collaboration and external sources are allowed and
  encouraged; see academic honesty policy for details.

\item[(30\%)] Research-oriented project and presentation.

\item[(20\%)] Participation (including scribe notes) and homework peer
  review.
\end{itemize}

All submitted work will be graded on \emph{correctness} and
\emph{clarity}, and must be typeset in \LaTeX\ (templates will be made
available).  It is good practice to start any longer solution with an
informal (but accurate) ``proof summary'' that describes the core idea
--- this will help the reader (and you!)  understand your solution
better.

There are no predetermined score thresholds for A/B/C/etc.  Your
primary focus should be on \emph{learning the material}, not your
grade.

\subsection{Academic Honesty}
\label{sec:academic-honesty}

On homework assignments, collaboration and consultation with external
sources is allowed and encouraged, subject to the following
conditions:
\begin{itemize}[itemsep=0pt]
\item You must submit your own individually written solution, and you
  must list your collaborators and/or external sources for each
  problem.
\item You may not submit a problem solution that you cannot explain
  orally.
\end{itemize}

There is no hard-and-fast list of (dis)honest conduct.  When in doubt,
err on the side of caution, or ask the instructor.  Dealing with
academic dishonesty is unpleasant for everyone involved, so please
follow these policies!

\end{document}

%%% Local Variables: 
%%% mode: latex
%%% TeX-master: t
%%% End: 
