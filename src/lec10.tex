\documentclass[11pt]{article}

\usepackage{fullpage}
\usepackage{newtxtext}
\usepackage{microtype}
\usepackage{amsmath,amsfonts,amssymb}
\usepackage[amsmath,amsthm,thmmarks,hyperref]{ntheorem}
\usepackage[pagebackref=true]{hyperref} % must set backref at load time
\usepackage{fancyhdr}
\usepackage{mathtools}
\usepackage{enumitem}
\usepackage[svgnames]{xcolor}
\usepackage{tikz}
\usepackage{import}

% load after hyperref, algpseudocode to get proper behavior
\usepackage[capitalize,nameinlink,noabbrev]{cleveref}

\import{common/}{head.tex}

% VARIABLES

\newcommand{\lecturenum}{10}
\newcommand{\lecturetopic}{Smoothing Parameter}
\newcommand{\scribename}{Hank Carter}

% END OF VARIABLES

\import{common/}{lechead.tex}
\lecheader                      % execute lecture commands

\pagestyle{plain}               % default: no special header

\begin{document}

\thispagestyle{fancy}           % first page should have special header

% LECTURE MATERIAL STARTS HERE


Here we introduce a fundamental lattice quantity called the
\emph{smoothing parameter}, which is essentially the amount of
Gaussian noise needed to completely ``smooth out'' the discrete
structure of a lattice.

\section{Background}
\label{sec:background}

Recall the following material from previous lectures.

\begin{definition}[Dual lattice]
  For a (full-rank) lattice $\lat \subset \R^{n}$, its \emph{dual
    lattice} is
  \[ \lat^{*} := \set{\vecw \in \R^{n} : \inner{\vecw,\lat} \subseteq
      \Z}. \]
\end{definition}

Last time we showed several basic facts about the dual lattice, e.g.,
$(\lat^{*})^{*} = \lat$, $\det(\lat^{*}) = \det(\lat)^{-1}$, and
$\matB$ is a basis of $\lat$ if and only if $\matB^{-t}$ is a basis of
$\lat^{*}$.

\begin{lemma}[Poisson Summation Formula (PSF)]
  \label{lem:psf}
  For any ``nice enough'' $f \colon \R^n \to \C$ having Fourier
  transform $\hat{f}$, we have
  \begin{equation}
    \label{eq:psf}
    f(\lat) = \det(\lat^*)\cdot \hat{f}(\lat^*).
  \end{equation}
\end{lemma}

\begin{lemma}[Periodization]
  \label{lem:periodization}
  For any full-rank lattice $\lat \subset \R^{n}$ and ``nice enough''
  function $f \colon \R^{n} \to \C$, the induced $\lat$-periodic
  function $g \colon \R^{n}/\lat \to \C$ defined as
  \[ g(\vecx+\lat) = f(\vecx+\lat) = \sum_{\vecv \in \vecx+\lat}
    f(\vecv) \] has Fourier series
  $\hat{g}(\vecw) = \det(\lat^{*}) \cdot \hat{f}(\vecw)$ for all
  $\vecw \in \lat^{*}$.
\end{lemma}

The \emph{Gaussian function} $\rho \colon \R^{n} \to \R^{+}$ is
defined as
\begin{equation}
  \label{eq:gaussian}
  \rho(\vecx) = \exp(-\pi \length{\vecx}^{2}) = \exp(-\pi \inner{\vecx,
    \vecx}),
\end{equation}
and its scaling with parameter $s > 0$ is
$\rho_{s}(\vecx) = \rho(\vecx/s) = \exp(-\pi
\length{\vecx}^{2}/s^{2})$. Recall that $\hat{\rho} = \rho$, and
$\hat{\rho_{s}} = s^{n} \cdot \rho_{1/s}$.

\section{Smoothing Parameter}
\label{sec:smoothing-parameter}

Last time, we considered the $\lat$-periodic Gaussian function
$\rho(\vecx+\lat)$, and showed that it can be ``lumpy,'' in the sense
that:
\begin{enumerate}
\item $\rho(\vecx+\lat) \geq \frac{1}{1000} \cdot \rho(\lat)$ when
  $\dist(\vecx,\lat) \leq 1$
\item $\rho(\vecx+\lat) \leq 2^{-n} \cdot \rho(\lat)$ when
  $\dist(t,\lat) \geq \sqrt{n}$, which is equivalent to
  $(\vecx+\lat) \cap \sqrt{n} \ball = \emptyset$.
\end{enumerate}

Today, we consider sufficient conditions that make this function
``smooth,'' i.e., nearly equal on every coset $\vecx+\lat$. First,
recall that by the shift rule for the Fourier transform and the PSF
(twice),
\begin{align}
  \rho(\vecx+\lat) 
  &= \det(\lat^*) \cdot \sum_{\vecw \in
    \lat^*} \rho(\vecw) \cdot \exp(2\pi i \inner{\vecw,\vecx})
    \label{eq:rho-coset} \\
  &\leq \det(\lat^*) \cdot \rho(\lat^{*}) = \rho(\lat), \nonumber
\end{align}  
where the inequality follows because
$\exp(2\pi i \inner{\vecw,\vecx}) \in \C$ is on the complex unit
circle and $\rho(\vecw) > 0$.

When is this upper bound essentially tight, i.e., when do we have
near-equality? The following important theorem gives a sufficient
condition.

\begin{theorem}
  \label{thm:smooth}
  If $\rho(\lat^* \setminus \set{\veczero}) \leq \varepsilon$ for some
  $\varepsilon > 0$, then
  $\rho(\vecx + \lat) \geq \frac{1-\varepsilon}{1+\varepsilon} \cdot
  \rho(\lat)$ for all $\vecx \in \R^{n}$.
\end{theorem}

\begin{proof}
  Continuing from \cref{eq:rho-coset}, and separating out the case
  $\vecw=\veczero$, we have
  \begin{align*}
    \rho(\vecx+\lat) 
    &= \det(\lat^*) \cdot (1 + \sum_{\vecw \in \lat^*
      \setminus \set{\veczero}} \rho(\vecw) \cdot \exp(2\pi i
      \inner{\vecw,\vecx})) \\
    &\geq \det(\lat^{*}) \cdot (1 - \sum_{\vecw \in \lat^{*} \setminus
      \set{0}} \rho(\vecw)) \\
    &\geq \det(\lat^{*}) \cdot (1-\varepsilon) \\
    &\geq \frac{1-\varepsilon}{1+\varepsilon} \cdot \det(\lat^{*}) \cdot
      \rho(\lat^{*}) \\
    &= \frac{1-\varepsilon}{1+\varepsilon} \cdot \rho(\lat),
  \end{align*} 
  where the first inequality (again) uses the fact that
  $\abs{\exp(2 \pi i \inner{\vecw,\vecx})} = 1$ and $\rho(\vecw) > 0$,
  and the second and third inequalities uses the hypothesis.
\end{proof}

\Cref{thm:smooth} says that if the dual lattice has small Gaussian
mass, then shifting the (primal) lattice has little effect on its
Gaussian mass. The mass is maximized on the lattice's zero coset, but
the mass changes only slightly as one shifts the lattice around. This
fact motivates the following definition.

\begin{definition}[Smoothing
  parameter~\cite{DBLP:journals/siamcomp/MicciancioR07}]
  For an $\varepsilon > 0$, the \emph{smoothing parameter}
  $\smooth_{\varepsilon}(\lat)$ of a lattice~$\lat$ is the smallest
  $s > 0$ such that
  $\rho_{1/s}(\lat^* \setminus \set{\veczero}) \leq \varepsilon$.
\end{definition}

Note that as~$s$ increases, $1/s$ decreases, so
$\rho_{1/s}(\lat^* \setminus \set{\veczero})$ decreases. So for any
$s \geq \smooth_{\varepsilon}(\lat)$, we have
$\rho_{1/s}(\lat^{*} \setminus \set{\veczero}) \leq \varepsilon$. This
in turn means that
\[ \rho_{s}(\vecx + \lat) \geq \frac{1-\varepsilon}{1+\varepsilon}
  \cdot \rho_{s}(\lat) \] for any $\vecx \in \R^{n}$, simply by
scaling and applying \cref{thm:smooth}.

The following small claim will be useful for bounding the smoothing
parameter by bounding the \emph{fraction} of Gaussian mass on the
non-zero points of the dual lattice.

\begin{claim}
  \label{clm:relative-smoothing}
  If
  $\rho_{1/s}(\lat^* \setminus \set{\veczero}) \leq
  \frac{\varepsilon}{1+\varepsilon} \cdot \rho_{1/s}(\lat^*)$, then
  $\rho_{1/s}(\lat^* \setminus \set{\veczero}) \leq \varepsilon$, and
  hence $s \geq \smooth_{\varepsilon}(\lat)$.
\end{claim}

\begin{proof}
  We have
  $\rho_{1/s}(\lat^*) = 1 + \rho_{1/s}(\lat^* \setminus
  \set{\veczero}) \leq 1 + \frac{\varepsilon}{1+\varepsilon} \cdot
  \rho_{1/s}(\lat^{*})$. Rearranging proves the claim.
\end{proof}

We can relate $\smooth_\varepsilon(\lat)$ to ``standard'' lattice
quantities. We first connect it to the minimum distance of the dual
lattice.

\begin{lemma}
  For any (full-rank) lattice $\lat \in \R^{n}$ and
  $\varepsilon = 4^{-n}$, we have that
  \[ \smooth_\varepsilon(\lat) \leq
    \frac{\sqrt{n}}{\lambda_1(\lat^*)} . \]
\end{lemma}

\begin{proof}
  For notational convenience, by scaling~$\lat$ (and its dual) we can
  assume that $s = \sqrt{n}/\lambda_{1}(\lat^{*}) = 1$, so
  $\lambda_{1}(\lat^{*}) = \sqrt{n}$. We then need to show that
  $\rho(\lat^* \setminus \set{\veczero}) \leq \varepsilon = 2^{-n}$.

  Recall from the previous lecture that the Gaussian mass of any
  lattice (coset) outside the radius-$\sqrt{n}$ ball is a tiny
  fraction of the lattice's total Gaussian mass:
  $\rho((\vecx+\lat^*) \setminus \sqrt{n}\ball) \leq 5^{-n} \cdot
  \rho(\lat^*)$.\footnote{We previously stated the claim with only
    a~$2^{-n}$ factor, but a~$5^{-n}$ factor can be obtained from the
    same proof and the fact that $\exp(3 \pi/4) > 10$.} Since
  $\lambda_1(\lat^*) \geq \sqrt{n}$, we have that
  \begin{align*}
    \rho(\lat^{*} \setminus \set{\veczero})
    &= \rho(\lat^{*} \setminus \sqrt{n} \ball) \\
    &\leq 5^{-n} \cdot \rho(\lat^{*}) \\
    &\leq \frac{1}{4^{n}+1} \cdot \rho(\lat^{*}).
  \end{align*}
  The lemma then follows from \cref{clm:relative-smoothing}.
\end{proof}

\begin{lemma}
  \label{lem:smoothing-lower}
  For any lattice~$\lat$ and $\varepsilon \leq 2\exp(-\pi)$, we have
  that $\smooth_\varepsilon(\lat) \geq 1/\lambda_1(\lat^*)$.
\end{lemma}

\begin{proof}
  Again, by scaling we can assume that $s = 1/\lambda_1(\lat^*) = 1$,
  so $\lambda_{1}(\lat^{*}) = 1$. Then
  $\rho(\lat^* \setminus \set{\veczero}) \geq 2 \exp(-\pi)$, by
  considering just the Gaussian mass on any shortest nonzero vector
  and its negation.
\end{proof}

We next relate the smoothing parameter $\smooth_{\varepsilon}(\lat)$
to quantities of the primal lattice~$\lat$ itself. For intuition,
consider the possibility that $\smooth(\lat) \leq \lambda_1(\lat)$,
which intuitively could hold in many cases. However, this is false in
general. As a counterexample, consider the lattice with basis
$(1,0), (0,100)$. While the sum of the lattice-centered Gaussians on
the $x$-axis is fairly ``smooth,'' the lattice points on the $y$-axis
are too far separated, making the Gaussians ``lumpy'' in that
direction.

This can be seen in another way---which corresponds to the actual
definition of the smoothing parameter---by looking at the dual
lattice, which has basis $(1,0), (0,1/100)$. The Gaussian mass of
nonzero dual points on the $x$-axis is very small, but the mass of
those on the $y$-axis is fairly large, because they are so tightly
spaced together. Indeed, \cref{lem:smoothing-lower} says that
$\smooth_{\varepsilon} \geq 1/\lambda_{1}(\lat^{*}) = 100 \gg
\lambda_{1}(\lat) = 1$ even for a moderately small
constant~$\varepsilon > 0$.

Below we will give a general smoothing-parameter bound which shows
that, in this particular example (and as one might expect), to get
``smoothness'' we need to use Gaussians of width approximately~$100$,
times small extra factor.

\begin{definition}
  \label{def:successive-minimum}
  The \emph{$i$th successive minimum} of a
  lattice~$\lat$ is defined as
  \begin{align*}
    \lambda_i(\lat)
    &= \min \set{r : (\lat \cap r\ball) \text{
      contains at least~$i$ linearly independent vectors}} \\
    &= \min \set{r : \dim(\spn(\lat \cap r\ball)) \geq i} .
  \end{align*}
\end{definition}
Thus, for a (full-rank) $n$-dimensional lattice,
$\lambda_1 \leq \lambda_2 \leq \cdots \leq \lambda_n$.

\begin{theorem}[\cite{DBLP:journals/siamcomp/MicciancioR07}]
  \label{thm:smoothing-lambda_n}
  For any $\varepsilon > 0$ and (full-rank) $n$-dimensional lattice
  $\lat$, we have
  \[ \smooth_\varepsilon(\lat) \leq
    \lambda_n(\lat)\cdot\sqrt{\ln(2n(1+1/\varepsilon))/\pi}
    = \lambda_{n}(\lat) \cdot O(\sqrt{\ln(2n/\varepsilon)}). \]
\end{theorem}

For example, if we use an exponentially small
$\varepsilon = \exp(-n)$, then
$\sqrt{\ln(2n/\varepsilon)} \approx \sqrt{n}$. Or, we can use an
inverse-quasipolynomial $\varepsilon = 1/n^{\log n}$ to get
$\sqrt{\ln(2n/\varepsilon)} \approx \log n$. To prove
\cref{thm:smoothing-lambda_n}, we will use a new discrete tail bound
for Gaussians.

\begin{definition}
  For a unit vector $\vecu \in \R^{n}$ and real $t \geq 0$, define the
  (open) \emph{halfspace}
  \[ H_{\vecu,t} = \set{\vecx \in \R^n : \inner{\vecx,\vecu} < t} . \]
  In words, $H_{\vecu,t}$ is the set of points in $\R^{n}$ that are at
  most~$t$ from the origin in the direction of~$\vecu$.
\end{definition}

\begin{lemma}[\cite{DBLP:journals/dcg/Banaszczyk95}]
  \label{lem:gaussian-halfspace}
  For any lattice~$\lat$, unit vector~$\vecu \in \R^{n}$, real
  $t \geq 0$, and $\vecx \in \R^n$, we have that
  \[ \rho((\vecx+\lat) \setminus H_{\vecu,t}) \leq \exp(-\pi t^2)
    \cdot \rho(\lat) . \]
\end{lemma}

The proof works somewhat similarly to the proof of the tail bound from
the previous lecture. That is, we consider multiplying the Gaussian
mass of the coset points that are \emph{outside} the halfspace by some
very large factor. Then we show that the resulting mass is upper
bounded by a not-too-large multiple of the Gaussian mass on the
lattice. The only way this can be possible is if the (original)
Gaussian mass outside the halfspace is a small fraction of the
lattice's total mass.

\begin{proof}
  Because $\inner{\vecv, \vecu} \geq t$ for every
  $\vecv \not\in H_{\vecu,t}$, and by completing the square, we have
  \begin{align*}
    \exp(2 \pi t^{2}) \cdot \rho((\vecx+\lat) \setminus H_{\vecu,t})
    &\leq \sum_{\vecv \in \vecx+\lat} \rho(\vecv) \cdot \exp(2\pi
      \inner{\vecv, t\vecu}) \\
    &= \sum_{\vecv \in \vecx+\lat} \exp(-\pi (\inner{\vecv,\vecv} -
      2 \inner{\vecv, t\vecu})) \\
    &= \sum_{\vecv \in \vecx+\lat} \exp(-\pi \inner{\vecv-t\vecu,
      \vecv-t\vecu}) \cdot \exp(\pi \inner{t \vecu, t \vecu}) \\
    &= \exp(\pi t^{2}) \cdot \sum_{\vecw \in \vecx-t\vecu + \lat}
      \exp(-\pi \inner{\vecw,\vecw}) \\
    &= \exp(\pi t^{2}) \cdot \rho(\vecx-t\vecu + \lat) \\
    &\leq \exp(\pi t^{2}) \cdot \rho(\lat) .
  \end{align*}
  Dividing both sides of the inequality by $\exp(2 \pi t^2)$ proves
  the claim.
\end{proof}

(An interesting fact related to \cref{lem:gaussian-halfspace} is that
a \emph{continuous} Gaussian has the same kind of tail behavior: the
tail probability outside $H_{\vecu,t}$ is upper bounded by
$\exp(-\pi t^2)$. This can be shown by a similar argument using
integrals.)

\begin{proof}[Proof of \cref{thm:smoothing-lambda_n}]
  By \cref{clm:relative-smoothing}, it suffices to show that
  $\rho_{1/s}(\lat^* \setminus \set{\veczero}) \leq
  \frac{\varepsilon}{1+\varepsilon} \cdot \rho(\lat^{*})$. As usual,
  by scaling the lattice we can assume that
  $s = \lambda_{n}(\lat) \cdot \sqrt{\ln(2n(1+1/\varepsilon))/\pi} =
  1$.

  Let $\vecv_{1}, \vecv_{2}, \ldots, \vecv_{n} \in \lat$ be such that
  $\length{\vecv_{i}} = \lambda_{i} := \lambda_{i}(\lat)$ for all~$i$.
  (Actually, it will be enough that
  $\length{\vecv_{i}} \leq \lambda_{n}$.) Observe that any
  $\vecw \in \lat^{*} \setminus \set{\veczero}$ has \emph{nonzero
    integer} inner product with at least one of the~$\vecv_{i}$,
  because they span~$\R^{n}$ and~$\vecw$ is in the dual lattice. So,
  $\abs{\inner{\vecw, \vecv_{i}}} \geq 1$, which implies that
  \[ \abs{\inner{\vecw, \vecu_{i}}} \geq 1/\length{\vecv_{i}} \geq
    1/\lambda_{n} = \sqrt{\ln(2n(1+1/\varepsilon))/\pi} , \] where
  $\vecu_{i} = \vecv_{i}/\length{\vecv_{i}}$ is the unit vector in the
  direction of~$\vecv_{i}$. Therefore, every
  $\vecw \in \lat^{*} \setminus \set{\veczero}$ is outside \emph{at
    least one} halfspace $H_{\pm \vecu_{i}, 1/\lambda_{n}}$, so we
  have
  \[
    \rho(\lat^{*} \setminus \set{\veczero}) \leq \sum_{i=1}^{n}
    \rho(\lat^{*} \setminus H_{\vecu_{i}, 1/\lambda_{n}}) +
    \rho(\lat^{*} \setminus H_{-\vecu_{i}, 1/\lambda_{n}}) .
  \]
  
  We complete the proof by invoking \cref{lem:gaussian-halfspace}: it
  says that each term in the above sum is at most
  \[ \exp(-\pi / \lambda_{n}^{2}) \cdot \rho(\lat^{*}) =
    \exp(-\ln(2n(1+1/\varepsilon))) \cdot \rho(\lat^{*}) =
    \frac{\varepsilon}{2n(1+\varepsilon)} \cdot \rho(\lat^{*}). \] The
  sum has~$2n$ such terms, so we have shown that
  $\rho(\lat^{*} \setminus \set{\veczero}) \leq
  \frac{\varepsilon}{1+\varepsilon} \cdot \rho(\lat^{*})$, as desired.
\end{proof}

\bibliography{common/lattices,common/crypto}
\bibliographystyle{common/alphaabbrvprelim}

\end{document}

%%% Local Variables: 
%%% mode: latex
%%% TeX-master: t
%%% End: 
