\documentclass[11pt]{article}

\usepackage{fullpage}
\usepackage{newtxtext}
\usepackage{microtype}
\usepackage{amsmath,amsfonts,amssymb}
\usepackage[amsmath,amsthm,thmmarks,hyperref]{ntheorem}
\usepackage[pagebackref=true]{hyperref} % must set backref at load time
\usepackage{fancyhdr}
\usepackage{mathtools}
\usepackage{enumitem}
\usepackage[svgnames]{xcolor}
\usepackage{tikz}
\usepackage{import}

% load after hyperref, algpseudocode to get proper behavior
\usepackage[capitalize,nameinlink,noabbrev]{cleveref}

\import{common/}{head.tex}

% VARIABLES

\newcommand{\lecturenum}{6}
\newcommand{\lecturetopic}{Algorithms for SVP, CVP}
\newcommand{\scribename}{Sam Kim}

% END OF VARIABLES

\import{common/}{lechead.tex}
\lecheader                      % execute lecture commands

\pagestyle{plain}               % default: no special header

\begin{document}

\thispagestyle{fancy} % first page should have special header

% LECTURE MATERIAL STARTS HERE

\section{The Shortest and Closest Vector Problems}
\label{sec:svp-cvp}

Recall the definition of the approximate Shortest Vector Problem. (The
exact version is obtained by taking $\gamma=1$, which is implicit
when~$\gamma$ is omitted.)

\begin{definition}
  \label{def:svp}
  For $\gamma = \gamma(n) \geq 1$, the $\gamma$-approximate Shortest
  Vector Problem $\svp_{\gamma}$ is: given a basis $\matB$ of a
  lattice $\lat = \lat(\matB) \subset \R^{n}$, find some nonzero
  $\vecv \in \lat$ such that
  $\length{\vecv} \leq \gamma(n) \cdot \lambda_{1}(\lat)$.
\end{definition}

\noindent A closely related inhomogeneous variant is the approximate
Closest Vector Problem.

\begin{definition}
  \label{def:cvp}
  For $\gamma = \gamma(n) \geq 1$, the $\gamma$-approximate Closest
  Vector Problem $\cvp_{\gamma}$ is: given a basis $\matB$ of a
  lattice $\lat = \lat(\matB) \subset \R^{n}$ and a point
  $\vect \in \R^{n}$, find some $\vecv \in \lat$ such that
  $\length{\vect - \vecv} \leq \gamma(n) \cdot \dist(\vect,\lat)$.
  Equivalently, find an element of the lattice coset $\vect + \lat$
  having norm at most $\gamma(n) \cdot \lambda(\vect+\lat)$, where
  $\lambda(\vect+\lat) := \min_{\vecx \in \vect+\lat} \length{\vecx} =
  \dist(\vect,\lat)$.
\end{definition}

Above we have used the fact that
$\dist(\vect, \lat) = \min_{\vecv \in \lat} \length{\vect - \vecv} =
\min_{\vecx \in \vect+\lat} \length{\vecx}$, because $\lat = -\lat$.
The two versions of $\cvp$ are equivalent by associating each
$\vecv \in \lat$ with $\vect - \vecv \in \vect + \lat$, and vice
versa. Although the former version of the problem is the more
``obvious'' formulation, the latter version is often more convenient
in algorithmic settings, so we will use it throughout these notes.

We first show that $\svp_{\gamma}$ is no harder than $\cvp_{\gamma}$;
more specifically, given an oracle for $\cvp_{\gamma}$ we can solve
$\svp_{\gamma}$ efficiently.

\begin{theorem}
  For any $\gamma \geq 1$, we have $\svp_{\gamma} \leq \cvp_{\gamma}$
  via a Cook reduction.
\end{theorem}

\begin{proof}
  Consider the following algorithm that, given a lattice basis
  $\matB=(\vecb_1, \ldots, \vecb_n)$, and $\cvp$ oracle $\Ora$,
  outputs some $\vecv \in \lat = \lat(\matB)$:
  \begin{itemize}[itemsep=0pt]
  \item For each $i=1,\ldots, n$, compute basis
    $\matB_i = (\vecb_1, \ldots, \vecb_{i-1}, 2\vecb_i, \vecb_{i+1},
    \ldots, \vecb_n )$ and let $\vecv_i = \Ora(\matB_i, \vecb_i)$.
  \item Output one of the $\vecv_i$ that has minimal length
    $\length{\vecv_{i}}$.
  \end{itemize}
  We claim that this algorithm solves $\svp_{\gamma}$, i.e., it
  returns some nonzero lattice vector of length at most
  $\gamma \cdot \lambda_{1}(\lat)$.

  First observe that for each $i$, the lattice
  $\lat_{i} = \lat(\matB_{i}) \subset \lat$ consists of all those
  vectors in $\lat$ whose $\vecb_{i}$-coordinate is even, whereas the
  coset $\vecb_{i} + \lat_{i} \subset \lat$ consists of all those
  whose $\vecb_{i}$-coordinate is odd. Therefore,
  $\veczero \not\in \vecb_{i} + \lat_{i}$ for all $i$, so
  $\lambda(\vecb_{i} + \lat_{i}) \geq \lambda_{1}(\lat)$. Moreover, if
  $\vecv \in \lat$ is any shortest nonzero lattice vector, then at
  least one of its coefficients with respect to $\matB$ must be odd,
  otherwise $\vecv/2 \in \lat$ would be a shorter nonzero lattice
  vector. Therefore, $\vecv \in \vecb_{i} + \lat_{i}$ for at least one
  $i$, and so $\lambda(\vecb_{i}+\lat_{i}) = \lambda_{1}(\lat)$ for
  all such $i$.

  Now by hypothesis on $\Ora$, for every $i$ we have
  $\vecv_{i} \in \vecb_{i} + \lat_{i} \subset \lat \setminus
  \set{\veczero}$ and
  $\length{\vecv_{i}} \leq \gamma \cdot \lambda(\vecb_{i} +
  \lat_{i})$. Since
  $\lambda(\vecb_{i} + \lat_{i}) = \lambda_{1}(\lat)$ for at least one
  $i$, some $\length{\vecv_{i}} \leq \gamma \cdot \lambda_{1}(\lat)$,
  and correctness follows.
\end{proof}

We note that essentially the same reduction works for the
\emph{decisional} variants $\gapsvp_{\gamma}$ and $\gapcvp_{\gamma}$
of the problems, where instead of returning vectors $\vecv_{i}$, the
oracle $\Ora$ returns yes/no answers, and the reduction outputs the
logical OR of all the answers.

\subsection{Algorithms for $\svp$ and $\cvp$}
\label{sec:algorithms-svp-cvp}

The following are some historical milestones in algorithms for $\svp$
and $\cvp$ (for simplicity, we ignore polynomial factors in the
dimension~$n$ and bit length of the input basis):
\begin{itemize}[itemsep=0pt]
\item Using the LLL algorithm~\cite{lenstra82:_factor} and brute-force
  search over coefficient vectors, one can get an algorithm that
  solves $\svp$ and $\cvp$ in $2^{O(n^{2})}$ time and $\poly(n)$
  space.
\item In 1983, Kannan~\cite{DBLP:conf/stoc/Kannan83} gave
  deterministic algorithms that solve $\svp$ and $\cvp$ in
  $n^{O(n)} = 2^{O(n \log n)}$ time and $\poly(n)$ space.
\item In 2001, Ajtai, Kumar, and Sivakumar
  (AKS)~\cite{DBLP:conf/stoc/AjtaiKS01} gave \emph{randomized}
  ``sieve'' algorithms that solve $\svp$ and $\cvp_{1+\epsilon}$ (for
  any constant $\epsilon > 0$) in singly exponential $2^{O(n)}$ time
  \emph{and} space. (For $\cvp_{1+\varepsilon}$, the exponent in the
  running time depends inversely on~$\varepsilon$.)
\item In 2010, Micciancio and Voulgaris
  (MV)~\cite{DBLP:conf/stoc/MicciancioV10} gave a \emph{deterministic}
  algorithm that solves $\cvp$ (and hence $\svp$ and other problems)
  in $2^{2n}$ time and $2^{n}$ space.
\item In 2015, Aggarwal, Dadush, Regev, and
  Stephens-Davidowitz~\cite{DBLP:conf/stoc/AggarwalDRS15} gave a
  randomized algorithm that solves $\svp$ in $2^{n}$ time and space
  (note that the exponent here is exactly one). A follow-up work by
  Aggarwal, Dadush, and
  Stephens-Davidowitz~\cite{DBLP:conf/focs/AggarwalDS15} obtained a
  similar result for $\cvp$.
\end{itemize}
It is an important open question whether there exists a singly
exponential-time (or better) algorithm that uses only polynomial
space, or even subexponential space.

\section{The Micciancio-Voulgaris Algorithm for $\cvp$}
\label{sec:mv-alg}

\newcommand{\Vbar}{\ensuremath{\bar{\mathcal{V}}}}

The MV algorithm solves $\cvp$ on any $n$-dimensional lattice in
$2^{O(n)}$ time (for simplicity, we ignore polynomial factors in the
input length). It is based around the (closed) \emph{Voronoi} cell of
the lattice, which, to recall, is the set of all points in $\R^{n}$
that are as close or closer to the origin than to any other lattice
point:
\begin{equation}
  \label{eq:voronoi}
  \Vbar(\lat) = \set{\vecx \in \R^n : \length{\vecx} \leq
    \length{\vecx - \vecv} \; \forall\;
    \vecv \in \lat \setminus \set{\veczero} }.
\end{equation}
We often omit the argument $\lat$ when it is clear from context. From
the definition it can be seen that for any coset $\vect+\lat$, the set
$(\vect+\lat) \cap \Vbar$ consists of all the shortest elements of
$\vect+\lat$. For any lattice point~$\vecv$, define the halfspace
\begin{align}
  \label{eq:Hv}
  H_{\vecv}
  &= \set{ \vecx : \length{\vecx} \leq \length{\vecx-\vecv}} \\
  &= \set{\vecx : 2\inner{\vecx, \vecv} \leq \inner{\vecv, \vecv}}.
\end{align}
It is easy to see that $\Vbar$ is the intersection of $H_{\vecv}$ over
all $\vecv \in \lat \setminus \set{\veczero}$. The minimal set $V$ of
lattice vectors such that $\Vbar = \bigcap_{\vecv \in V} H_{\vecv}$ is
called the set of \emph{Voronoi-relevant} vectors; we often call them
\emph{relevant} vectors for short. The following characterizes the
relevant vectors of a lattice:

\begin{fact}[Voronoi]
  \label{fact:voronoi}
  A nonzero lattice vector $\vecv \in \lat \setminus \set{\veczero}$
  is relevant if and only if $\pm \vecv$ are the only shortest vectors
  in the coset $\vecv + 2\lat$.
\end{fact}

\begin{corollary}
  An $n$-dimensional lattice has at most $2(2^{n}-1) \leq 2^{n+1}$
  relevant vectors.
\end{corollary}

\begin{proof}
  Every relevant vector belongs to some nonzero coset of $\lat/2\lat$,
  of which there exactly $2^{n}-1$. By the above, there are at most
  two relevant vectors in each such coset.
\end{proof}

The MV algorithm has the following high-level structure. Given a basis
$\matB$ of a lattice $\lat = \lat(\matB)$ and a target point $\vect$
defining a coset $\vect + \lat$, the algorithm works as follows:
\begin{enumerate}
\item Compute a description of the Voronoi cell $\Vbar(\lat)$, as a
  list containing all the relevant vectors $\vecv \in \lat$, of which
  there are at most $2^{n+1} = 2^{O(n)}$. (The possibly exponential
  number of relevant vectors is the sole reason for the algorithm's
  exponential space complexity.)\label{item:compute-voronoi}

  Note that this ``preprocessing'' phase depends only on the lattice
  $\lat$, not the target $\vect$, so it can be performed before the
  target is known, and the result can be reused for additional
  targets.

\item Use the relevant vectors to ``walk,'' starting from $\vect$,
  through a sequence of vectors in the coset $\vect+\lat$ by adding
  appropriately chosen lattice vectors to the target. The walk finally
  terminates at a point in $(\vect+\lat) \cap \Vbar$, which is a
  solution to the $\cvp$ instance.\label{item:walk}

  The walk proceeds in phases, where each phase starts with some
  $\vect_{k} \in (\vect+\lat) \cap 2^{k} \cdot \Vbar$, and outputs
  some $\vect_{k-1} \in (\vect + \lat) \cap 2^{k-1} \cdot \Vbar$. We
  show below that each phase takes $2^{O(n)}$ time, and by using LLL
  we can ensure that the initial target point $\vect$ is in
  $2^{O(n)} \cdot \Vbar$, so the total number of phases is only
  $O(n)$. Therefore, the overall runtime of this step is $2^{O(n)}$.
\end{enumerate}

\subsection{The Walk}
\label{sec:walk}

We first describe how \cref{item:walk}, the ``walk,'' is
performed. The first observation is that by a scaling argument, it
suffices to show how to perform the final phase of the walk, which
takes some $\vect_{2} \in (\vect+\lat) \cap 2 \Vbar$ and outputs some
$\vect_{1} \in (\vect+\lat) \cap \Vbar$. Then all prior phases can be
performed using this procedure with a suitable scaling of the lattice:
since $2^{k} \cdot \Vbar(\lat) = 2 \cdot \Vbar(2^{k-1} \lat)$, we can
use the procedure on the lattice $2^{k-1} \lat$ (whose relevant
vectors are just $2^{k-1}$-factor scalings of $\lat$'s relevant
vectors) to go from some
$\vect_{k} \in (\vect+\lat) \cap 2^{k} \cdot \Vbar(\lat)$ to some
$\vect_{k-1} \in (\vect+\lat) \cap 2^{k-1} \cdot \Vbar(\lat)$.

The walk from $2 \Vbar$ to $\Vbar$ works as follows: first, check if
our current target $\vect \in \Vbar$, by testing if
$\vect \in H_{\vecv}$ for all $\vecv \in V$; if so, then simply output
$\vect$. Otherwise, subtract an appropriately chosen relevant
vector~$\vecv \in V$ from $\vect$ and loop. The main question is,
which relevant vector should we subtract to ensure that we make
progress, and terminate within $2^{O(n)}$ iterations?

Recall that if $\vect \not\in \Vbar(\lat)$, then it lies outside the
halfspace $H_{\vecv}$, i.e., it violates the inequality
$2\inner{\vect, \vecv} \leq \inner{\vecv, \vecv}$, for some relevant
vector $\vecv$. The MV algorithm greedily chooses a relevant vector
$\vecv$ whose inequality is ``most violated,'' i.e., it maximizes the
ratio $2 \inner{\vect, \vecv}/\inner{\vecv, \vecv}$, and subtracts
$\vecv$ from $\vect$. Observe that for such~$\vecv$, if we define
$\alpha = 2\inner{\vect, \vecv}/\inner{\vecv,\vecv}$, then $\alpha$ is
the smallest positive real number such that
$\vect \in \alpha \Vbar(\lat)$; by assumption, we have
$1 < \alpha \leq 2$.

\begin{lemma}
  \label{lem:MV-steps}
  The walk from $\vect_{2} \in (\vect+\lat) \cap 2\Vbar$ to
  $\vect_{1} \in (\vect+\lat) \cap \Vbar$ terminates within at most
  $2^{n}$ iterations.
\end{lemma}

The above lemma follows by combining two lemmas that we state and
prove next. The first shows that subtracting the chosen $\vecv$ brings
the target closer to the origin, while staying within the same
multiple of the Voronoi cell.

\begin{lemma}
  \label{lem:MV-progress}
  For any $\vect \notin \Vbar$, if $\vecv \in \lat$ is a relevant
  vector maximizing
  $\alpha = 2 \inner{\vect, \vecv}/\inner{\vecv, \vecv}$, then
  $\vect - \vecv \in \alpha \Vbar$ and
  $\length{\vect - \vecv} < \length{\vect}$.
\end{lemma}

\begin{proof}
  Below we show that $\vect$ and $\vect-\alpha \vecv$ have equal
  norms, and hence that both lie in $\alpha \Vbar$. Then because
  $\alpha > 1$, it follows that $\vect-\vecv$ is on the interior of
  the line segment between $\vect$ and $\vect - \alpha \vecv$, so
  $\length{\vect - \vecv} < \length{\vect}$ and
  $\vect - \vecv \in \alpha \Vbar$ by convexity of $\Vbar$, as
  claimed.

  Because $\alpha = 2 \inner{\vect, \vecv}/\inner{\vecv, \vecv}$, we
  have
  \begin{equation}
    \length{\vect}^{2} = \inner{\vect,\vect} =
    \inner{\vect, \vect} - 2 \alpha \inner{\vect, \vecv} + \alpha^{2} \inner{\vecv, \vecv}
    = \length{\vect- \alpha \vecv}^{2}. 
  \end{equation}
  Now because $\vect \in \alpha \Vbar(\lat) = \Vbar(\alpha \lat)$, it
  must be that $\vect$ is a shortest element of $\vect + \alpha\lat$.
  Since $\alpha \vecv \in \alpha \lat$, we also have
  $\vect - \alpha \vecv \in \vect + \alpha \lat$. Then because
  $\length{\vect} = \length{\vect - \alpha \vecv}$, we conclude that
  $\vect-\alpha\vecv$ is also a shortest element in
  $\vect + \alpha \lat$, and so
  $\vect-\alpha \vecv \in \alpha \Vbar(\lat)$.

  Finally, because $\alpha > 1$ and $\vecv \neq \veczero$ we have
  \[\length{\vect-\vecv} = \length{\vect}^2 + \length{\vecv}^2 - 2
    \inner{\vect, \vecv} = \length{\vect}^2 - (\alpha-1)
    \length{\vecv}^2 < \length{\vect}^2. \]
\end{proof}

\noindent
The second lemma bounds the number of distinct lengths our
intermediate target vectors can have.

\begin{lemma}
  \label{lem:distinct-lengths}
  For any $\vect$, let $U = (\vect + \lat) \cap 2\Vbar(\lat)$. Then
  $\abs*{ \set*{ \length{\vecu} : \vecu \in U}} \leq 2^n$.
\end{lemma}

\begin{proof}
  For any $\vect' \in \R^n$, the points in
  $(\vect' + 2\lat) \cap 2\Vbar (\lat)$ are the shortest vectors in
  the coset $\vect' + 2\lat$, and therefore all have the same length.
  Because $\lat$ is the union of $2^{n}$ distinct cosets
  $\vect' + 2\lat$, we see that $\vect + \lat$ is also the union of
  $2^n$ cosets of $2\lat$. By partitioning the vectors in $U$
  according to these cosets, we conclude that these vectors have at
  most $2^n$ distinct lengths overall.
\end{proof}

We can now prove \cref{lem:MV-steps}: since each step strictly
decreases the length of the target while keeping it in~$2\Vbar$, and
the intermediate targets can take on at most $2^{n}$ distinct lengths
overall, the walk must terminate within $2^{n}$ steps. This completes
the analysis of the ``walk'' step.

\subsection{Computing the Voronoi Cell}
\label{sec:comp-voron-cell}

We only summarize the main ideas behind the computation of the
Voronoi-relevant vectors of the lattice $\lat=\lat(\matB)$. The basic
idea is the compute them in a ``bottom-up'' fashion, by iteratively
computing the relevant vectors of the lower-rank lattices
$\lat_{1} = \lat(\vecb_{1})$, $\lat_{2} = \lat(\vecb_{1}, \vecb_{2})$,
$\lat_{3}=\lat(\vecb_{1}, \vecb_{2}, \vecb_{3})$, etc. Clearly, the
relevant vectors of $\lat_{1} = \lat(\vecb_{1})$ are just
$\set{\pm \vecb_{1}}$.

To iteratively compute the relevant vectors of~$\lat_{i}$, we use a
$\cvp$ oracle for~$\lat_{i-1}$, which we can implement using the
``walk'' step with the already-computed relevant vectors of
$\lat_{i-1}$. Here we use \cref{fact:voronoi}, which says that
every relevant vector $\vecv \in \lat_{i}$ is the (unique, up to sign)
shortest vector of some coset $\vect + 2\lat_{i}$ for
$\vect \in \lat_{i}$. So if we find a shortest vector of each of the
$2^{n}$ such cosets $\vect + 2\lat_{i}$, we will find every relevant
vector. (We might find other non-relevant vectors as well, but these
do not interfere with the ``walk'' step, so they can safely be
retained. In fact, there is a way to check for relevance if desired.)

For each coset $\vect + 2\lat_{i}$, we find a shortest element using
our $\cvp$ oracle for $\lat_{i-1}$. This is done by partitioning the
coset $\vect + 2\lat_{i}$ according to the $\vecb_{i}$ coefficient,
which yields several ``slices'' that each correspond to some coset of
$2\lat_{i-1}$. By using an LLL-reduced basis $\matB$ we can ensure
that the number of slices we need to inspect is only $2^{O(n)}$. For
each of the slices we find a shortest element in the corresponding
coset, and then take a shortest one overall to get a shortest element
of $\vect + 2\lat_{i}$.

Overall, to find the relevant vectors of $\lat_{i}$ given those of
$\lat_{i-1}$, we need to solve $\cvp$ on $2^{n}$ cosets of
$2\lat_{i}$, each of which reduces to solving $\cvp$ on $2^{O(n)}$
cosets of $2\lat_{i-1}$, each of which takes $2^{O(n)}$ time using the
``walk'' step. So the relevant vectors of $\lat_{i}$ can be computed
in $2^{O(n)}$ time overall, and hence so can the relevant vectors of
$\lat = \lat(\matB)$.

\bibliography{common/lattices,common/crypto}
\bibliographystyle{common/alphaabbrvprelim}

\end{document}

%%% Local Variables: 
%%% mode: latex
%%% TeX-master: t
%%% End: 
