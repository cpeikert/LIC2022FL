\documentclass[11pt]{article}

\usepackage{fullpage}
\usepackage{newtxtext}
\usepackage{microtype}
\usepackage{amsmath,amsfonts,amssymb}
\usepackage[amsmath,amsthm,thmmarks,hyperref]{ntheorem}
\usepackage[pagebackref=true]{hyperref} % must set backref at load time
\usepackage{fancyhdr}
\usepackage{mathtools}
\usepackage{enumitem}
\usepackage[svgnames]{xcolor}
\usepackage{tikz}
\usepackage{import}

\usepackage{algorithm}
\usepackage[noend]{algpseudocode}
\renewcommand{\algorithmicrequire}{\textbf{Input:}}
\renewcommand{\algorithmicensure}{\textbf{Output:}}

% load after hyperref, algpseudocode to get proper behavior
\usepackage[capitalize,nameinlink,noabbrev]{cleveref}

\import{common/}{head.tex}
\import{common/slides/}{tikzhead.tex}
\definecolor{structure.fg}{rgb}{0.2,0.2,0.7} % from favorite beamer theme

% VARIABLES

\newcommand{\lecturenum}{3}
\newcommand{\lecturetopic}{LLL, Coppersmith}
\newcommand{\scribename}{Yan Wang}

% END OF VARIABLES

\import{common/}{lechead.tex}
\lecheader                      % execute lecture commands

\pagestyle{plain}               % default: no special header

\begin{document}

\thispagestyle{fancy} % first page should have special header

% LECTURE MATERIAL STARTS HERE

\section{The LLL Algorithm}
\label{sec:lll-algorithm}

Recall the definition of an LLL-reduced lattice basis, as defined by
Lenstra, Lenstra, and Lov{\'a}sz~\cite{lenstra82:_factor}, and how it
approximates a short nonzero lattice vector. (In all that follows,
recall that $\matB = \gs{\matB} \cdot \matU$ is the Gram-Schmidt
decomposition of a lattice basis~$\matB$, where the columns
of~$\gs{\matB}$ are orthogonal and $\matU \in \R^{n \times n}$ is
upper unitriangular, with entries~$\mu_{i,j}$.)

\begin{definition}
  \label{def:lll}
  A lattice basis~$\matB$ is \emph{LLL-reduced} if the following hold:
  \begin{enumerate}
  \item For all $i < j$, we have
    $\abs{\mu_{i,j}} \leq \frac{1}{2}$.\label{item:size}
    % 
    \hfill (Such a basis is said to be ``sized reduced.'')
    
  \item For all $i < n$, we have
    $\frac{3}{4}\length{ \gs{\vecb}_i}^2 \leq
    \length{\mu_{i,i+1}\gs{\vecb}_i +
      \gs{\vecb}_{i+1}}^2$.\label{item:lovasz}
    %
    \hfill (This is the ``Lov{\'a}sz condition.'')
  \end{enumerate}
\end{definition}

\begin{lemma}
  \label{lem:lll-gs}
  In an LLL-reduced lattice basis $\matB \in \Z^{n \times n}$, we have
  $\length{\gs{\vecb}_{i+1}}^{2} \geq \frac12
  \length{\gs{\vecb}_{i}}^{2}$ for all $i < n$, and hence
  $\length{\vecb_1} \leq 2^{(n-1)/2} \min_{i} \length{\gs{\vecb}_{i}}
  \leq 2^{(n-1)/2} \cdot \lambda_{1}(\lat(\matB))$.
\end{lemma}

\cref{alg:lll} defines the LLL algorithm, which transforms any lattice
basis into an LLL-reduced one of the same lattice. The algorithm
simply alternates between two steps: \emph{size-reducing} the current
basis (i.e., making \cref{item:size} hold), and checking whether the
Lov{\'a}sz condition (\cref{item:lovasz}) holds; if not, it simply
swaps a pair of offending basis vectors. From this description, we can
immediately see that the output is indeed LLL-reduced, so the
algorithm is correct---if it ever terminates. The key challenge is in
showing that it does indeed terminate, and in time polynomial in the
input size.

\begin{algorithm}
  \caption{The LLL algorithm.}
  \label{alg:lll}

  \begin{algorithmic}[1]
    \Function{\algo{LLL}}{$\matB \in \Z^{n \times n}$}
    \Comment{Outputs an LLL-reduced basis of the lattice
      $\lat(\matB)$}

    \State Let
    $\matB \gets \algo{SizeReduce}(\matB)$\label{step:size-reduce}

    \If{there exists an $i < n$ for which the Lov{\'a}sz condition is
      violated} \State swap~$\vecb_{i}$ and~$\vecb_{i+1}$ and go to
    \cref{step:size-reduce} \Else\ output $\matB$\label{step:swap}
    \EndIf

    \EndFunction

    \Function{\algo{SizeReduce}}{$\matB \in \Z^{n \times n}$}
    \Comment{Outputs a size-reduced basis of $\lat(\matB)$, preserving
      $\gs{\matB}$}

    \State Compute (or update) the Gram-Schmidt
    orthogonalization~$\gs{\matB}$ of~$\matB$\label{step:compute-gs}

    \For{$j = 2, \ldots, n$ (in any order)}
    \For{$i = j-1 \text{ down to } 1$}\label{step:i-downto}

    \State let
    $\vecb_{j} \gets \vecb_{j} - \round{u_{i,j}} \cdot \vecb_{i}$,
    where
    $u_{i,j} = \inner{\vecb_{j},
      \gs{\vecb}_{i}}/\inner{\gs{\vecb}_{i}, \gs{\vecb}_{i}}$

    \EndFor \EndFor \EndFunction
  \end{algorithmic}
\end{algorithm}

The $\algo{SizeReduce}(\matB)$ subroutine deserves some elaboration.
Its purpose is, in the Gram-Schmidt decomposition
$\matB = \gs{\matB} \cdot \matU$, to shift the entries in the upper
triangle of~$\matU$ by integers, so that they lie in $\hohalf$.
Because we must preserve the lattice, this is done by subtracting
appropriate integer multiples of the basis vectors~$\vecb_{i}$
(\emph{not} the Gram-Schmidt vectors~$\gs{\vecb}_{i}$!) from
the~$\vecb_{j}$, for $i < j$. In matrix form, the $(i,j)$th iteration
lets $\matB \gets \matB \cdot \matW = \gs{\matB} \cdot (\matU \matW)$,
where $\matW$ is the upper unitriangular matrix with just one possibly
nonzero off-diagonal entry $-\round{u_{i,j}}$, at position $(i,j)$.
This implicitly updates the (upper unitriangular) matrix
$\matU \gets \matU \matW \in \R^{n \times n}$. This matrix is
identical to~$\matU$, except possibly at position $(i,j)$ \emph{and
  the entries above it, but not below it} in the same column. This is
why we need to make the changes going \emph{upward} in each column
(see \cref{step:i-downto}).

\begin{lemma}
  \label{lem:size-reduce-correct}
  Given an integral lattice basis $\matB \in \Z^{n \times n}$ with
  Gram-Schmidt orthogonalization $\gs{\matB}$, the $\algo{SizeReduce}$
  algorithm outputs a basis~$\matB'$ of $\lat = \lat(\matB)$ having
  Gram-Schmidt decomposition $\matB' = \gs{\matB} \cdot \matU'$, where
  $u'_{i,j} \in \hohalf$ for all $i<j$.
\end{lemma}

\begin{proof}
  First, even though $\algo{SizeReduce}$ may change~$\matB$, the
  Gram-Schmidt vectors~$\gs{\matB}$ are preserved throughout, because
  the only changes to~$\matB$ are via multiplication by
  upper-unitriangular matrices, i.e., if
  $\matB = \gs{\matB} \cdot \matU$ is the Gram-Schmidt decomposition
  prior to some iteration, then
  $\matB = \gs{\matB} \cdot (\matU \matW)$ is the decomposition
  afterward, since $\matU \matW$ is upper unitriangular. Second, the
  $(i,j)$th iteration ensures that the value
  $\inner{\vecb_{j}, \gs{\vecb}_{i}}/\inner{\gs{\vecb}_{i},
    \gs{\vecb}_{i}} \in \hohalf$, and that value never changes in
  later iterations, because for all $h < i$, the basis
  vector~$\vecb_{h}$ (a multiple of which may later be subtracted
  from~$\vecb_{j}$) is orthogonal to~$\gs{\vecb}_{i}$.
\end{proof}

\noindent We now state the main theorem about the LLL algorithm.

\begin{theorem}
  \label{thm:lll}
  Given an integral lattice basis $\matB \in \Z^{n \times n}$, the LLL
  algorithm outputs an LLL-reduced basis of $\lat = \lat(\matB)$ in
  time $\poly(n, \len{\matB})$, where $\len{\matB}$ denotes the bit
  length of the input basis.
\end{theorem}

The remainder of this section is dedicated to an (almost complete)
proof of this theorem. First, it is clear that the LLL algorithm, if
it ever terminates, is correct: all the operations on the input basis
preserve the lattice it generates, and the algorithm can only output
an LLL-reduced basis.

We next prove that the number of iterations is $O(N)$ for some
$N=\poly(n, \len{\matB})$. This uses a ``potential argument,'' which
assigns a value to all the intermediate bases produced by the
algorithm. We show three facts: that the potential starts out no
larger than~$2^{N}$, that it never drops below~$1$, and that each
iteration of the algorithm decreases the potential by a factor of at
least $\sqrt{4/3} > 1$. All this implies that the number of iterations
is at most $\log_{\sqrt{4/3}} 2^{N} = O(N)$.

The potential function is defined as follows: for a basis
$\matB = (\vecb_{1}, \ldots, \vecb_{n})$, let
$\lat_{i} = \lat(\vecb_{1}, \ldots, \vecb_{i})$ for each $i \leq n$.
The potential is the product of the determinants of these lattices:
\begin{equation}
  \label{eq:potential}
  \Phi(\matB) := \prod_{i=1}^{n} \det(\lat_{i}) = \prod_{i=1}^{n}
  \parens*{\length{\gs{\vecb}_{1}} \cdots \length{\gs{\vecb}_{i}}} =
  \prod_{i=1}^{n} \length{\gs{\vecb}_{i}}^{n-i+1}.
\end{equation}

\begin{claim}
  \label{clm:potential}
  The potential of the input basis~$\matB$ is at most~$2^{N}$, and
  every intermediate basis the algorithm produces has potential at
  least~$1$.
\end{claim}

\begin{proof}
  Because $\length{\gs{\vecb}_{i}} \leq \length{\vecb_{i}}$  and
  $\length{\vecb_{i}} \geq 1$ for
  all~$i$, the potential of the input basis~$\matB$ is bounded by
  \[ \Phi(\matB) \leq \prod_{i=1}^{n} \length{\vecb_{i}}^{n-i+1} \leq
    \prod_{i=1}^{n} \length{\vecb_{i}}^{n} \leq \max_{i}
    \length{\vecb_{i}}^{n^{2}} = 2^{\poly(n,\len{\matB})}. \] Every
  intermediate basis is integral, hence the lattices~$\lat_{i}$
  associated with that basis have determinant at
  least~$1$.\footnote{This requires some care to verify, because the
    lattices~$\lat_{i}$ are \emph{not full rank} for $i < n$. Letting
    $\matB_{i} = (\vecb_{1}, \ldots, \vecb_{i}) \in \Z^{n \times i}$
    be a basis for~$\lat_{i}$, we have
    $\det(\lat_{i}) = \sqrt{\det(\matB_{i}^{t} \matB_{i})} \geq 1$,
    because the Gram matrix
    $\matB_{i}^{t} \matB_{i} \in \Z^{i \times i}$ is integral and
    nonsingular.} Therefore, the potential of that basis is at
  least~$1$.
\end{proof}

\noindent We next analyze how the potential changes when we perform a
swap in \cref{step:swap}.

\begin{claim}
  \label{clm:swap-gs}
  Suppose that~$\vecb_{i}$ and~$\vecb_{i+1}$ are swapped in
  \cref{step:swap}, and let the resulting basis be denoted~$\matB'$.
  Then $\gs{\vecb}'_{j} = \gs{\vecb}_{j}$ for all
  $j \not\in \set{i,i+1}$, and
  $\gs{\vecb}'_{i} = \mu_{i,i+1} \gs{\vecb}_{i} + \gs{\vecb}_{i+1}$.
\end{claim}

Note that this claim lets us optimize \cref{step:compute-gs}:
following a swap, we need not orthogonalize~$\matB'$ from scratch, but
only need to update two of the vectors from~$\gs{\matB}$.

\begin{proof}
  For all $j < i$, we have $\gs{\vecb}'_{j} = \gs{\vecb}_{j}$, because
  it is the component of $\vecb'_{j} = \vecb_{j}$ orthogonal to
  $\spn(\vecb'_{1}, \ldots, \vecb'_{j-1}) = \spn(\vecb_{1}, \ldots,
  \vecb_{j-1})$. Similarly, for any $j > i+1$, we have
  $\gs{\vecb}'_{j} = \gs{\vecb}_{j}$, because it is the component of
  $\vecb'_{j} = \vecb_{j}$ orthogonal to
  $\spn(\vecb'_{1}, \ldots, \vecb'_{j-1}) = \spn(\vecb_{1}, \ldots,
  \vecb_{j-1})$, where here the equality holds because
  both~$\vecb_{i}$ and~$\vecb_{i+1}$ are in the lists of arguments (in
  reversed order). Finally, $\gs{\vecb}'_{i}$ is the component of
  $\vecb'_{i} = \vecb_{i+1}$ orthogonal to
  $\spn(\vecb'_{1}, \ldots, \vecb'_{i-1}) = \spn(\vecb_{1}, \ldots,
  \vecb_{i-1})$, which is
  $\mu_{i,i+1} \gs{\vecb}_{i} + \gs{\vecb}_{i+1}$ by construction.
\end{proof}

\begin{lemma}
  \label{lem:lll-potential}
  Suppose that~$\vecb_{i}$ and~$\vecb_{i+1}$ are swapped in
  \cref{step:swap}, and let the resulting basis be denoted~$\matB'$.
  Then $\Phi(\matB')/\Phi(\matB) < \sqrt{3/4}$.
\end{lemma}

\begin{proof}
  Let $\lat_i = \lat( \vecb_1, \ldots , \vecb_{i-1}, \vecb_i ) $ and
  $\lat_i' = \lat( \vecb_1, \ldots , \vecb_{i-1}, \vecb_{i+1} ) $. By
  \cref{clm:swap-gs}, we have
  \begin{equation}
    \begin{split}
      \frac{\Phi(\matB')}{\Phi(\matB)} =
      \frac{\det(\lat_i')}{\det(\lat_i)} =
      \frac{\length{\gs{\vecb}_{1}} \cdots \length{\gs{\vecb}_{i-1}}
      \length{\mu_{i,i+1} \gs{\vecb}_{i} + \gs{\vecb}_{i+1}}
      }{\length{\gs{\vecb}_{1}} \cdots \length{\gs{\vecb}_{i-1}}
      \length{\gs{\vecb}_{i}}} = \frac{\length{\mu_{i,i+1}
      \gs{\vecb}_{i} + \gs{\vecb}_{i+1}}}{\length{\gs{\vecb}_{i}}}
      < \sqrt{3/4},
    \end{split}
  \end{equation}
  where the last inequality follows from the fact that the swap occurs
  because the Lov{\'a}sz condition (\cref{item:lovasz}) is \emph{not}
  satisfied.
\end{proof}

This completes the proof that the number of iterations is
$O(N) = \poly(n, \len{\matB})$. Moreover, we can see by inspection
that each iteration of the algorithm is polynomial time in the bit
length of the current basis. However, this does \emph{not} necessarily
guarantee that the LLL algorithm is polynomial time overall, since the
bit length of the intermediate bases could possibly \emph{increase}
with each iteration. (For example, if the bit length doubled in each
iteration, then by the end, the bit length would be exponential
in~$n$.)

To complete the full proof that LLL is polynomial time, it suffices to
show that the sizes of \emph{all} intermediate bases are some fixed
polynomial in the size of the original basis. This turns out to be the
case, specifically due to the size-reduction step (which we have not
used up to this point!). The proof of this fact is somewhat grungy,
though, so we won't cover it.

We conclude with some final remarks about the LLL algorithm. The
factor $3/4$ in the Lov{\'a}sz condition is just for convenience of
analysis. We can use any constant between $1/4$ and $1$, which yields
a tradeoff between the final approximation factor and the number of
iterations, but these will still remain exponential and polynomial
(in~$n$), respectively. By choosing a value very close to~$1$, we can
obtain an approximation factor of $(2/\sqrt{3})^{n}$ in polynomial
time, but we cannot do any better using LLL.\@ We can get slightly
better approximation factors of $2^{O(n(\log\log n)^2/\log n)}$,
still in polynomial time, using Schnorr's
generalization~\cite{DBLP:journals/tcs/Schnorr87} of LLL, where the
analogue of the Lov{\'a}sz condition deals with blocks of $k \geq 2$
consecutive vectors.

\section{Coppersmith's Method}
\label{sec:coppersmiths-method}

One nice application of LLL is a technique of
Coppersmith~\cite{DBLP:conf/eurocrypt/Coppersmith96} that finds all
\emph{small} roots of a polynomial modulo a given number~$N$, even
when the factorization of $N$ is unknown. This technique has been a
very powerful tool in cryptanalysis, as we will see next time.

\begin{theorem}
  \label{thm:coppersmith}
  There is a polynomial-time algorithm that, given any monic,
  degree-$d$ integer polynomial $f(x) \in \Z[x]$ and an integer $N$,
  outputs all integers $x_{0}$ such that
  $\abs{x_{0}} \leq B = N^{1/d}$ and $f(x_{0}) = 0 \bmod N$.
\end{theorem}

\noindent We make a few important remarks about the various components
of this theorem:
\begin{enumerate}[itemsep=0pt]
\item When $N$ is prime, i.e., $\Z_{N}$ is a finite field, there are
  efficient algorithms that output \emph{all} roots of a given
  degree-$d$ polynomial $f(x)$ modulo $N$, of which there are at most
  $d$. Similarly, there are efficient algorithm that factor
  polynomials over the rationals (or integers). Therefore, the fact
  that the theorem handles a composite modulus $N$ is a distinguishing
  feature.
\item For composite $N$, the number of roots of $f(x)$ modulo~$N$ can
  be nearly exponential in the bit length of~$N$, even for quadratic
  $f(x)$. For example, if~$N$ is the product of~$k$ distinct primes,
  then any square modulo~$N$ has exactly~$2^{k}$ distinct square
  roots. (This follows from the Chinese Remainder Theorem, since there
  are two square roots modulo each prime divisor of $N$.) Since~$k$
  can be as large as $\approx \log N / \log \log N$, the number of
  roots can be nearly exponential in $\log N$. Therefore, in general
  no efficient algorithm can output \emph{all} roots of $f(x)$
  modulo~$N$; the restriction to \emph{small} roots in the theorem
  statement circumvents this problem.\footnote{Indeed, the theorem
    implies that the number of small roots is always polynomially
    bounded. This fact did not appear to be known before Coppersmith's
    result!}
\item The size restriction appears necessary for another reason:
  knowing two square roots $r_{1} \neq \pm r_{2}$ of a square modulo a
  composite~$N$ reveals a nontrivial factor of~$N$, as
  $\gcd(r_{1}-r_{2}, N)$. So even if the number of roots is small,
  finding all of them is still at least as hard as factoring. However,
  it is easy to show that a square cannot have more than one ``small''
  square root, of magnitude at most $N^{1/2}$. Therefore, the theorem
  does not appear to yield an efficient factoring algorithm. (However,
  it can be used to factor when some partial information about a
  factor is known, as shown in the related
  work~\cite{DBLP:conf/eurocrypt/Coppersmith96a}.)
\end{enumerate}

To highlight the heart of the method, in the remainder of this section
we prove the theorem for a weaker bound of $B \approx N^{2/(d(d+1))}$,
where the approximation hides a small constant factor. (We will prove
the theorem for $B \approx N^{1/d}$ next time.) The strategy is to
find another nonzero polynomial $h(x) = \sum h_{i} x^{i} \in \Z[x]$
such that:
\begin{enumerate}[itemsep=0pt]
\item every root of $f(x)$ modulo $N$ is also a root of $h(x)$, and
\item the polynomial $h(Bx)$ is ``short,'' i.e.,
  $\abs{h_{i} B^{i}} < N/(\deg(h)+1)$ for all $i$.
\end{enumerate}
For any such $h(x)$, and for any $x_{0}$ such that $|x_0| \leq B$, we
have $\abs{h_i x_0^i} \leq \abs{h_{i} B^{i}} < N/(\deg(h)+1)$, which
implies that $\abs{h(x_0)} < N$. Hence, for every \emph{small}
root~$x_0$ (such that $\abs{x_{0}} \leq B$) of $f(x)$ modulo~$N$, we
have that $h(x_0)=0$ \emph{over the integers} (not modulo anything).
To find the small roots of $f(x)$ modulo~$N$, we can therefore just
factor $h(x)$ over the integers, and test whether each of its (small)
roots is a root of $f(x)$ modulo~$N$.

We now give an efficient algorithm to find such an $h(x)$. The basic
idea is that adding integer multiples of the polynomials
$g_{i}(x) = N x^i \in \Z[x]$ to $f(x)$ certainly preserves the roots
of~$f$ modulo~$N$. So we construct a lattice whose basis corresponds
to the coefficient vectors of the polynomials $g_{i}(Bx)$ and $f(Bx)$,
find a short nonzero vector in this lattice, and interpret it as the
polynomial $h(Bx)$. The lattice basis is
\[ \matB =
  \begin{pmatrix}
    N & & & & f_0 \\
      & B N & & & f_1 B \\
      & & B^2 N & & f_2 B^2   \\
      & & & & \\
      & & & B^{d-1} N & f_{d-1} B^{d-1} \\
      & & & & B^d
  \end{pmatrix}. \]
Note that the lattice dimension is $d+1$, and that
$\det(\matB) = B^{d(d+1)/2} \cdot N^d$. By running the LLL algorithm
on this basis, we obtain a nonzero lattice vector~$\vecv \in
\lat(\matB)$ for which
\[ \length{\vecv} \leq 2^{d/2} \cdot B^{d/2} \cdot N^{d/(d+1)} =
  (2B)^{d/2} \cdot N^{1-1/(d+1)}.
\]

Define $h(Bx)$ to be the polynomial whose coefficients are given by
$\vecv$, i.e., $h(x) = v_0 + (v_{1}/B)x + \cdots + (v_{d}/B^{d}) x^d$.
Notice that $h(x) \in \Z[X]$, because $B^i$ divides $v_i$ for each $i$
by construction of the lattice basis, and that every root of $f(x)$
modulo $N$ is also a root of $h(x)$ by construction. Finally, we see
that
\[ \abs{ h_{i} B^{i} } = \abs{v_{i}} \leq \length{v} <
  \frac{N}{d+1}, \] if we take $(2B)^{d/2} < N^{1/(d+1)}/(d+1)$, which
is equivalent to $B < N^{2/(d(d+1))}/(2 (d+1)^{2/d})$; note that the
denominator is upper bounded by a constant. This concludes the proof.

\bibliography{common/lattices}
\bibliographystyle{common/alphaabbrvprelim}

\end{document}

%%% Local Variables: 
%%% mode: latex
%%% TeX-master: t
%%% End: 
