\documentclass[11pt]{article}

\usepackage{fullpage}
\usepackage{newtxtext}
\usepackage{microtype}
\usepackage{amsmath,amsfonts,amssymb}
\usepackage[amsmath,amsthm,thmmarks,hyperref]{ntheorem}
\usepackage[pagebackref=true]{hyperref} % must set backref at load time
\usepackage{fancyhdr}
\usepackage{mathtools}
\usepackage{enumitem}
\usepackage[svgnames]{xcolor}
\usepackage{tikz}
\usepackage{import}

% load after hyperref, algpseudocode to get proper behavior
\usepackage[capitalize,nameinlink,noabbrev]{cleveref}

\import{common/}{head.tex}

% VARIABLES

\newcommand{\lecturenum}{12}
\newcommand{\lecturetopic}{SIS Lattices \& Applications}
\newcommand{\scribename}{Jacob Alperin-Sheriff}

% END OF VARIABLES

\import{common/}{lechead.tex}
\lecheader                      % execute lecture commands

\newcommand{\calR}{\ensuremath{\mathcal{R}}}
\newcommand{\calI}{\ensuremath{\mathcal{I}}}

\pagestyle{plain}               % default: no special header

\begin{document}

\thispagestyle{fancy} % first page should have special header

% LECTURE MATERIAL STARTS HERE

\section{SIS Lattices}
\label{sec:sis-lat}

In this lecture we give properties and applications of SIS (short
integer solution) lattices. We first recall the SIS problem.

\begin{definition}[Shortest Integer Solution Problem]
  \label{def:sis-problem}
  For a positive integer modulus~$q$, dimensions~$n,m$ and a norm
  bound~$\beta > 0$, the $\sis_{n,q,\beta,m}$ problem is defined as
  follows: given uniformly random $\matA \in \Zq^{n \times m}$, find a
  nonzero ``short'' solution $\vecz \in \Z^{m}$ such that
  $\matA \vecz = \veczero \in \Zq^{n}$ and
  $\length{\vecz} \leq \beta$.

  Equivalently, the goal is to find a non-zero vector of norm at
  most~$\beta$ in the following integer ``SIS lattice'' (it is easy to
  verify that this set is a discrete additive subgroup):
  \[ \latperp(\matA) := \set{\vecz \in \Z^{m} : \matA\vecz = \veczero}
    . \] Borrowing a term from coding theory, matrix~$\matA$ is often
  called a \emph{parity-check matrix} for the lattice
  $\latperp(\matA)$.
\end{definition}

We begin with some mathematical properties of SIS lattices. First,
these lattices are called ``$q$-ary'' because they contain every
integer vector whose entries are all multiples of~$q$:
\[
  q\Z^{m} \subseteq \latperp(\matA) \subseteq \Z^{m}.
\]
So, a vector's membership in $\latperp(\matA)$ is determined solely by
its entries modulo~$q$.

\paragraph{Cosets.}

Borrowing another term from coding theory let $\vecy \in \Zq^{n}$ be a
``syndrome'' in the image of~$\matA$, i.e., there exists some
$\vecx \in \Z^{m}$ such that $\matA \vecx = \vecy$. Then we can define
the corresponding lattice coset
\[
  \latperp_{\vecy}(\matA) := \set{ \vecx' \in \Z^m : \matA \vecx' =
    \vecy } = \vecx + \latperp(\matA) , \] where the equality holds
because every $\vecx' \in \vecx + \latperp(\matA)$ satisfies
$\matA \vecx' = \matA \vecx = \vecy$, and for any $\vecx' \in \Z^{m}$
such that $\matA \vecx' = \vecy$, we have
$\vecx' = \vecx + (\vecx' - \vecx) \in \vecx + \latperp(\matA)$
because $\matA (\vecx'-\vecx) = \veczero$.

\paragraph{Determinant.}

By the bijective correspondence between integer cosets and syndromes
in the image of~$\matA$, we have that
\[ \det(\latperp(\matA)) = \abs{\Z^m / \latperp(\matA)} =
  \abs{\text{Image}(\matA)} \leq q^{n} , \] with equality if the image
of~$\matA$ is all of~$\Zq^{n}$ (i.e., the columns of~$\matA$
generate~$\Zq^{n}$), which will be the setting we are almost always
interested in.

If~$q$ is prime, then $\matA \in \Zq^{n \times m}$ generates all
of~$\Zq^n$ if and only if it contains an $n \times n$ submatrix that
is invertible modulo~$q$. (It is not too hard to show that when, say,
$m \geq (1+\delta)n$ for a constant $\delta > 0$, this holds with high
probability over the uniformly random choice of~$\matA$.) However, if
$q$ is composite, this condition is sufficient but not necessary. For
example, $\matA=(2, 3) \in \Z_6^{1 \times 2}$ generates all of~$\Z_6$,
even though neither~$2$ nor~$3$ are invertible modulo~$6$.

\paragraph{Minimum distance.}

The minimum distance can be bounded using Minkowski's Theorem and the
above bound on the determinant:
\begin{align*}
  \lambda_1(\latperp(\matA))
  &\leq \sqrt{m} \cdot \det(\latperp(\matA))^{1/m} \\
  &\leq \sqrt{m} \cdot q^{n/m} .
\end{align*}
However, observe that we are not required to use all~$m$ dimensions:
by fixing some of the vectors' coordinates to zero (equivalently,
dropping some columns of~$\matA$), we can instead work with a lattice
of any dimension less than~$m$. Assuming that the original dimension
is any $m = \Omega(n \log q)$, the above bound is optimized for some
dimension $\Theta(n \log q)$, and yields a minimum distance of
$O(\sqrt{n \log q})$.

Also recall from last lecture that if $m > n \log_{2} q$, then there
exists a nonzero $\set{0,\pm 1}^{m}$-vector in $\latperp(\matA)$, so
the~$\ell_{p}$ minimum distance is
$\lambda_{1}^{(p)}(\latperp(\matA)) \leq m^{1/p}$ for any finite~$p$,
and $\lambda_{1}^{(\infty)}(\latperp(\matA)) \leq 1$.

% text early end of document
\end{document}

\paragraph{Equivalent representations and lattices.}

We observe that many different matrices~$\matA$ can define
(essentially) the same lattice. The proofs of the following two lemmas
are straightforward exercises.

\begin{lemma}
  \label{lem:parity-check-invertible}
  Let $\matH \in \Zq^{n \times n}$ be invertible. Then
  \[ \latperp(\matH \cdot \matA) = \latperp(\matA) . \]
\end{lemma}

\begin{lemma}
  \label{lem:parity-check-permutation}
  Let~$\matA'$ be a column permutation of~$\matA$, i.e.,
  $\matA'=\matA\matP$, where $\matP \in \bit^{m \times m}$ is a
  permutation matrix. Then
  \[ \latperp(\matA') = \matP^{-1} \cdot \latperp(\matA) , \] i.e.,
  the lattice $\latperp(\matA')$ is just a coordinate permutation of
  the lattice $\latperp(\matA)$, so it has the same determinant,
  successive minima, etc.
\end{lemma}

More generally, if $\matT \in \Zq^{m \times m}$ is invertible, then
\[ \latperp(\matA \cdot \matT) = \matT^{-1} \cdot \latperp(\matA)
  \pmod{q}. \] However, for general~$\matT$ this is not merely a
coordinate permutation, so it may change the lattice's essential
geometric properties.

\paragraph{(Canonical) basis.}

Suppose that~$\matA$ contains an $n \times n$ submatrix that is
invertible modulo~$q$. Without loss of generality, by permuting
columns (\cref{lem:parity-check-permutation}), we can write
$\matA=[\matH \mid \matA']$ where $\matH=\Zq^{n \times n}$ is
invertible. Then by \cref{lem:parity-check-invertible}, without
changing the lattice we can write the parity-check matrix as
$[\matI_{n} \mid \bar{\matA}]$ where
$\bar{\matA} = \matH^{-1} \matA' \in \Zq^{n \times (m-n)}$. Borrowing
another coding theory term, this is called the ``systematic form,'' or
more commonly ``(Hermite) normal form'' in the lattice context.
Observe that it is a slightly more compact representation, because we
can treat the identity submatrix as implicit.

We then have the following basis for the lattice
$\lat = \latperp([\matI_{n} \mid \bar{\matA}]) \subseteq \Z^{m}$:
\[ \matB =
  \begin{bmatrix}
    q\matI_n & -\bar{\matA} \\
    \matzero & \matI_{m-n}
  \end{bmatrix} \in \Z^{m \times m}.
\]
(Formally, because $-\bar{\matA}$ is a matrix over~$\Zq$ (not~$\Z$),
the basis uses any integer-matrix representative of it, e.g., with
entries in $\set{0,1,\ldots,q-1}$.) We can confirm that~$\matB$ is
indeed a basis of~$\lat$:
\begin{itemize}
\item First, the columns are linearly independent, because $\matB$ is
  upper triangular with nonzero diagonal entries.
\item Second, every column vector of~$\matB$ is in the lattice,
  because $[\matI_{n} \mid \bar{\matA}] \cdot \matB = \matzero$. But
  do these columns form a \emph{basis} of~$\lat$?
\item We have $\det(\matB) = q^{n} = \det(\lat)$, so~$\matB$ is in
  fact a basis.
\end{itemize}

For a (full-rank) integer lattice, a basis of the above form is said
to be in ``Hermite normal form.'' The precise requirements are that
the basis be upper triangular with positive entries on the diagonal,
and that every other entry is reduced modulo the diagonal entry of its
row. It turns out that such a basis is uniquely specified for any
lattice, and is efficiently computable from any basis of the same
lattice. Therefore, the Hermite normal form basis acts as a kind of
``canonical'' representation of an integer lattice.
(See~\cite{DBLP:conf/issac/MicciancioW01,DBLP:conf/calc/Micciancio01}
for further details.)

\section{Applications of SIS}
\label{sec:applications}

\subsection{Collision-Resistant Hashing}
\label{sec:coll-resist-hash}

\begin{definition}
  A function $f \colon \calD \to \calR$ where
  $\abs{\calD} > \abs{\calR}$ is called a \emph{hash function}. A set
  $\calF = \set{f_{a} \colon \calD \to \calR}$ of such functions is
  called a \emph{hash function family}.
\end{definition}

Since the domain~$\calD$ is larger than the range~$\calR$, by the
pigeonhole principle any function $f \colon \calD \to \calR$ must have
a \emph{collision}, i.e., some distinct $x_1, x_2 \in \calD$ where
$f_a(x_1)=f_a(x_2)$. A hash function family is said to be
\emph{collision resistant} if it is infeasible to find a collision in
a randomly chosen function from the family. To make this asymptotic,
we parameterize the family (and the domain and range) by a security
parameter $\lambda \in \N$.

\begin{definition}
  A hash function family~$\calF_{\lambda}$ is \emph{collision
    resistant} if for any efficient adversary $\Adv$,
  \[ \Pr_{f_a \gets \calF_{\lambda}}[\Adv(f_a) \text{ outputs a
      collision in $f_{a}$}] = \negl(\lambda) . \]
\end{definition}

Collision resistance is a central property in cryptography. It is
often used to ``compress'' data into a small digest, so that only the
original data can be presented as consistent with that digest (because
otherwise a collision has been found, which is infeasible). This can
be used for detecting errors or modifications in data, authenticating
arbitrary pieces of large data sets, and much more.

We can construct a collision-resistant hash function family directly
from the $\sis$ problem.
\begin{theorem}
  Let $n=\lambda$ (the security parameter), $q \geq 2$ be arbitrary,
  $m > n \log_{2} q$, and $\beta=\sqrt{m}$. If $\sis_{n,q,\beta,m}$ is
  hard, then the $\sis$ function family
  $\calF=\set{f_{\matA} \colon \bit^{m} \to \Zq^{n} : \matA \in \Zq^{n
      \times m}}$, defined as
  $f_{\matA}(\vecx):=\matA\vecx \in \Zq^{n}$, is a collision-resistant
  hash function family.
\end{theorem}

We note that the domain need not consist merely of \emph{binary}
strings; we can generalize to vectors of ``small'' integers with a
suitable adjustment of the parameters.

\begin{proof}
  First, since $\abs{\calD}=2^{m} > q^{n} = \abs{\calR}$, we indeed
  have a family of hash functions. Let~$\Adv$ be an efficient
  adversary that attempts to break the collision resistance of the
  hash function family. We describe an efficient reduction that
  attempts to solve $\sis$ using~$\Adv$. Given an $\sis$ instance
  $\matA \in \Zq^{n \times m}$, the reduction gives~$\matA$ to the
  adversary~$\Adv$, which potentially outputs a collision
  in~$f_{\matA}$, i.e., distinct $\vecx, \vecx' \in \bit^{m}$ such
  that $\matA\vecx=\matA\vecx'$. The reduction outputs
  $\vecz = \vecx-\vecx' \in \set{0, \pm 1}^{m}$ as its potential
  $\sis$ solution.

  Suppose that~$\Adv$ was successful in outputting a collision. Then
  $\vecz \neq \veczero$,
  $\matA\vecz = \matA\vecx - \matA\vecx' = \veczero$, and
  $\length{\vecz} \leq \sqrt{m} = \beta$, so the reduction
  successfully solves its given $\sis$ instance. Because $\sis$ is
  hard by assumption, the adversary~$\Adv$ must have only negligible
  probability of success, as desired.
\end{proof}

\subsection{Regular One-Way Function Family}
\label{sec:regular-one-way}

\begin{definition}
  \label{def:regular}
  A function family $\calF = \set{f \colon \calD \to \calR}$ is
  \emph{$\varepsilon$-regular} if
  \[ (f, f(x)) \approx_{\varepsilon} (f, y) , \] where
  $f \gets \calF, x \gets \calD, y \gets \calR$,
  and~$\approx_{\varepsilon}$ denotes that the two distributions have
  statistical distance at most~$\varepsilon$.
\end{definition}

\begin{definition}
  \label{def:one-way}
  A function family
  $\calF_{\lambda} = \set{f_{a} \colon \calD_{\lambda} \to
    \calR_{\lambda}}$ is \emph{one way} if for all efficient
  adversaries~$\Adv$,
  \[\Pr_{f_a \gets \calF_{\lambda}, y \gets \calR_{\lambda}}[\Adv(f_a,
    y) \text{ outputs } x \in \calD_{\lambda} \text{ s.t. } f_a(x)=y]
    = \negl(\lambda) . \]
\end{definition}

In particular, regularity (for tiny $\varepsilon \geq 0$) implies that
for a randomly chosen function~$f_{a}$ from the family, almost every
range value $y \in \calR$ has a preimage, i.e., an $x \in \calD$ for
which $f_{a}(x)=y$. One-wayness can be seen as saying that finding
such a pre-image is hard.

\begin{lemma}%
  \label{lem:sis-regular}
  For $n \geq 1$, any $q \geq 2$, and any
  $m \geq (1+\delta) n \log_{2} q$ for constant $\delta > 0$, the
  $\sis$ family
  $\calF=\set{f_{\matA} \colon \bit^{m} \to \Zq^{n} \colon \matA \in
    \Zq^{n \times m}}$ is $2^{-\Omega(n)}$-regular.
\end{lemma}

\begin{lemma}%
  \label{lem:sis-oneway}
  The above $\sis$ function family is one-way if
  $\sis_{n,q,\beta,m+1}$ is hard, where $\beta=\sqrt{m+1}$.
\end{lemma}

\begin{proof}
  Let~$\Adv$ be an efficient adversary that attempts to break the
  one-wayness of the family. We describe an efficient
  reduction~$\AdvB$ that attempts to solve~$\sis_{n,q,\beta,m+1}$
  using~$\Adv$. The reduction works as follows: given an instance
  $\matA' \in \Zq^{n \times (m+1)}$, $\AdvB$ parses it as
  $\matA'=[\matA \mid \vecy]$ where $\vecy \in \Zq^{n}$. Then~$\AdvB$
  invokes $\calI(f_{\matA},\vecy)$, which potentially outputs some
  $\vecx \in \bit^{m}$ such that $\matA\vecx=\vecy$. Then~$\AdvB$
  outputs $\vecz = \smlmat{\vecx \\ -1} \in \Z^{m+1}$ as a potential
  solution to the $\sis$ instance.

  Observe that whenever~$\AdvA$ successfully inverts, we have
  $\matA\vecx=\vecy$ and hence
  $\matA' \vecz = \matA \vecx - \vecy = \veczero$; moreover,
  $\vecz \neq \veczero$ $\length{\vecz} \leq \sqrt{m+1}=\beta$,
  so~$\AdvB$ succeeds at solving its $\sis$ instance. Because $\sis$
  is hard by assumption,~$\Adv$ must have only negligible probability
  of success, as needed.
\end{proof}

\Cref{lem:sis-regular,lem:sis-oneway} together yield \emph{the central
  method} (up to some minor variations) of constructing secret/public
key pairs in lattice-based cryptography, which goes back to Ajtai's
original work~\cite{ajtai04:_gener_hard_instan_lattic_probl}. The
procedure operates as follows:
\begin{enumerate}[itemsep=0pt]
\item Generate (or be given, from a trusted source) a uniformly random
  matrix $\matA \in \Zq^{n \times m}$.
\item Choose a secret key $\vecx \gets \bit^{m} = \calD$ uniformly at
  random.
\item Let $\vecy = \matA\vecx = f_{\matA}(\vecx)$ and let the public
  key be $\matA' = [\matA \mid \vecy] \in \Zq^{n \times (m+1)}$ (or
  just~$\vecy$, if others already know~$\matA$).
\end{enumerate}
We can even do the same with multiple vectors~$\vecx_{i}$, which we
group into the columns of a secret-key matrix
$\matX \in \bit^{m \times k}$, yielding the public-key matrix
$\matA' = [\matA \mid \matY] \in \Zq^{n \times (m+k)}$ where
$\matY = \matA \matX \in \Zq^{n \times k}$.

The procedure above has the following properties:
\begin{itemize}[itemsep=0pt]
\item $\matA'$ is statistically close to uniformly random, by
  regularity (\cref{lem:sis-regular}).
\item Therefore, the adversary cannot find any ``short''
  $\vecz \in \latperp(\matA')$, by the presumed hardness of~$\sis$.
\item However, the generating party \emph{does} know such a short
  vector, namely, $\vecz = \smlmat{\vecx \\ -1}$.
\end{itemize}
Looking ahead, this asymmetry between the user and the adversary will
allow us to build more advanced cryptographic primitives, like digital
signatures and public-key encryption, among many others.

\begin{proof}[Proof of \cref{lem:sis-regular}]
  This follows immediately from
  \cref{lem:sis-universal,lem:universal-regular} below.
\end{proof}

\begin{definition}
  \label{def:universal}
  A function family $\calF = \set{f_{a} \colon \calD \to \calR}$ is
  \emph{universal} if for all distinct $x, x' \in \calD$,
  \[ \Pr_{f_a \gets \calF}[f_a(x)=f_a(x')] = \frac{1}{\abs{\calR}}
    . \]
\end{definition}

\begin{lemma}
  \label{lem:sis-universal}
  The $\sis$ function family with domain $\calD=\bit^{m}$ is
  universal.
\end{lemma}

\begin{proof}
  Fix arbitrary distinct $\vecx, \vecx' \in \bit^{m}$. Without loss of
  generality (by rearranging coordinates and swapping $\vecx,\vecx'$
  if necessary), we have that $x_1-x'_1 = 1$. Then
  \[ \Pr_{\matA}[\matA\vecx = \matA\vecx'] =
    \Pr_{\matA}[\matA(\vecx-\vecx') = \veczero] = \Pr_{\matA}[\veca_1
    = \sum_{i=2}^{m} \veca_{i} (x'_{i}-x_{i})] = q^{-n} =
    \frac{1}{\abs{\calR}} ,
  \]
  where the second-to-last equality follows by arbitrarily
  fixing~$\veca_{i}$ for $i=2,\ldots,m$, and considering the
  probability over~$\veca_{1}$ alone.
\end{proof}

\begin{lemma}[Leftover Hash Lemma~\cite{DBLP:journals/siamcomp/HastadILL99}]
  \label{lem:universal-regular}
  Any universal hash family~$\calF$ of functions from~$\calD$
  to~$\calR$ is $\varepsilon$-regular for
  $\varepsilon=\sqrt{\abs{\calR}/{\abs{\calD}}}$.
\end{lemma}

\begin{proof}
  Because the first components of the two distributions in
  \cref{def:regular} are a uniformly random function $f \gets \calF$,
  the statistical distance in question is simply the expected (i.e.,
  average) statistical distance between $f(x)$ and~$y$ (where
  $x \gets \calD$ and $y \gets \calR$), over the choice of~$f$.
  
  For any fixed hash function $f \in \calF$, let~$D_{f}$ be the
  distribution of $f(x)$. The squared~$\ell_{2}$ norm of this
  distribution is
  \[ \length{D_{f}}_{2}^{2} := \sum_{y \in \calR} D_{f}(y)^{2} =
    \Pr_{x,x' \gets \calD}[f(x)=f(x')] \leq \frac{1}{\abs{\calD}} +
    \Pr_{x,x'}[f(x)=f(x') \mid x \neq x'] . \] Taking the expectation
  over the uniformly random choice of $f \in \calF$, and using
  universality, we have that
  \[ \E_{f \gets \calF} \bracks[\big]{\length{D_{f}}_{2}^{2}} \leq
    \frac{1}{\abs{\calD}} + \frac{1}{\abs{\calR}} . \] So, by the
  relation between the~$\ell_{2}$ and~$\ell_{1}$ norms, the fact that
  the values of each~$D_{f}$ sum to unity, and Jensen's inequality,
  the expected statistical distance between~$D_{f}$ and the uniform
  distribution is
  \begin{align*}
    \E_{f \gets \calF} \bracks[\Big]{\sum_{y \in \calR} \abs[\big]{D_{f}(y)
    - \frac{1}{\abs{\calR}}}}
    &\leq \E_{f} \bracks[\Big]{ \abs{\calR}^{1/2} \cdot
      \parens[\Big]{\sum_{y \in \calR}
      \parens[\big]{D_{f}(y) - \frac{1}{\abs{\calR}}}^{2}}^{1/2}} \\
    &= \abs{\calR}^{1/2} \cdot \E_{f} \bracks[\Big]{ \parens[\Big]{
      \sum_{y \in \calR} D_{f}(y)^{2} - \frac{1}{\abs{\calR}}}^{1/2}} \\
    &\leq \abs{\calR}^{1/2} \cdot \E_{f}
      \bracks[\Big]{\length{D_{f}}_{2}^{2} -
      \frac{1}{\abs{\calR}}}^{1/2} \\
    &\leq \sqrt{\abs{\calR}/\abs{\calD}} .
  \end{align*}
\end{proof}

\bibliography{common/lattices,common/crypto}
\bibliographystyle{common/alphaabbrvprelim}

\end{document}

%%% Local Variables: 
%%% mode: latex
%%% TeX-master: t
%%% End: 
