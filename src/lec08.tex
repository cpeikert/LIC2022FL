\documentclass[11pt]{article}

\usepackage{fullpage}
\usepackage{newtxtext}
\usepackage{microtype}
\usepackage{amsmath,amsfonts,amssymb}
\usepackage[amsmath,amsthm,thmmarks,hyperref]{ntheorem}
\usepackage[pagebackref=true]{hyperref} % must set backref at load time
\usepackage{fancyhdr}
\usepackage{mathtools}
\usepackage{enumitem}
\usepackage[svgnames]{xcolor}
\usepackage{tikz}
\usepackage{import}

% load after hyperref, algpseudocode to get proper behavior
\usepackage[capitalize,nameinlink,noabbrev]{cleveref}

\import{common/}{head.tex}

% VARIABLES

\newcommand{\lecturenum}{8}
\newcommand{\lecturetopic}{Real Fourier Analysis}
\newcommand{\scribename}{Prateek Bhakta}

% END OF VARIABLES

\import{common/}{lechead.tex}
\lecheader                      % execute lecture commands

\pagestyle{plain}               % default: no special header

\begin{document}

\thispagestyle{fancy}           % first page should have special header

% LECTURE MATERIAL STARTS HERE

\section{Fourier Transform}
\label{sec:ftrans}

We begin with some basic definitions.

\begin{definition}[$L^1$ function]
  \label{def:L1}
  The function class $L^1(\R)$ is the set of all functions
  $f \colon \R \to \C$ for which
  \[ \int_{\R} \abs{f(x)} dx < \infty . \]
\end{definition}

\begin{definition}[Fourier transform]
  \label{def:fourier-transform}
  Given $f \in L^1(\R)$, its Fourier Transform
  $\hat{f} \colon \R \to \C$ is defined as
  \[\hat{f}(w) := \int_{\R} f(x) e^{-2\pi i x w} dx . \]
\end{definition}

\noindent
Note that
\[\abs{\hat{f}(w)} \leq \int_{\R} \abs{f(x)e^{-2\pi ixw}} dx
  \leq \int_{\R} \abs{f(x)} \cdot \abs{e^{-2\pi i x w}} dx = \int_{\R}
  \abs{f(x)}dx < \infty\] for all $w \in \R$, thus $\hat{f}(w)$ is
finite for all $w$. (The second to last step uses the fact that
$\abs{e^{-2\pi ixw}} = 1$.)

\begin{example}
  \label{ex:fourier-gaussian}
  The Fourier transform of the Gaussian function $f(x) = e^{-\pi x^2}$
  is
  \begin{align*}
    \hat{f}(w) 
    &= \int_{\R} e^{-\pi x^2} \cdot e^{-2\pi ixw} dx \\
    &= \int_{\R} e^{-\pi ((x + i w)^2 + w^2)} dx \\
    &= e^{-\pi w^2}\int_{\R} e^{-\pi (x + i w)^2} dx \\
    &= e^{-\pi w^2} \int_{\R} e^{-\pi x^2} dx
      {\footnotemark}\\
    &= e^{-\pi w^2} {\footnotemark} \\
    &= f(w),
  \end{align*}
  i.e., the Gaussian is its own Fourier transform.
\end{example}
 
\addtocounter{footnote}{-1} \footnotetext{This step follows from
  applying Cauchy's integral formula to the contour integral around
  the closed path $R \to R - i w \to -R - i w \to -R \to R$, as
  $R \to \infty$. Since $e^{-\pi x^2}$ is ``nice enough''
  (holomorphic), this integral must be 0. As $R \to \infty$, the
  integrals from $[R \to R - i w]$ and $[-R - i w \to -R]$ are each
  bounded in absolute value by $e^{-\pi w R^2}$, hence they both
  approach~$0$. Therefore,
  \[0 = \lim_{R \to \infty} \int_{-R}^{R} e^{-\pi x^2}dx + \int_{R - i
      w}^{-R - i w} e^{-\pi x^2}dx = \lim_{R \to \infty} \int_{-R}^{R}
    e^{-\pi x^2} dx + \int_{R}^{-R} e^{-\pi (u + i w)^2} du.\] The
  result follows by swapping the bounds of integration in the second
  integral, and taking $R \to \infty$. }

\stepcounter{footnote} \footnotetext{This follows from integrating
  the square in polar coordinates:
  \[ \int_{\R} e^{-\pi x^2} dx = \sqrt{\int_{\R^2} e^{-\pi x^2}e^{-\pi
        y^2}dx } = \sqrt{\int_{\theta=0}^{2\pi} \int_{r=0}^{\infty}
      e^{-\pi r^2} \cdot rdr d\theta} = \sqrt{\int_{\theta=0}^{2\pi}
      \frac{-1}{2\pi}(0-1)} = 1 . \]}

\paragraph{Basic facts.}

We recall some important facts about the Fourier transform.

\begin{fact}[Linearity]
  \label{fact:linear}
  For all $f,g \in L^1(\R)$ and all $a \in \R$,
  $\widehat{f + g} = \hat{f} + \hat{g}$ and
  $\widehat{a \cdot f} = a\hat{f}$.
\end{fact}

\begin{fact}[Time shift]
  \label{fact:shift}
  Let $f \in L^{1}(\R)$ and $h(x) = f(x - c)$ for some $c \in \R$.
  Then $\hat{h}(w) = e^{-2\pi i c w} \cdot \hat{f}(w)$:
  \[ \hat{h}(w) = \int_{\R} f(x-c)e^{-2\pi i xw}dx = \int_{\R} f(u)
    e^{-2\pi i(u + c)w} du = e^{-2\pi i c w} \cdot \hat{f}(w) . \]
\end{fact}

\begin{fact}[Time scale]
  \label{fact:scale}
  Let $f \in L^{1}(\R)$ and $h(x) = f(x/s)$ for some $s > 0$. Then
  $\hat{h}(w) = s \hat{f}(s w)$:
  \[ \hat{h}(w) = \int_{\R} f(x/s) e^{-2\pi i x w} dx = s
    \int_{\R} f(u) e^{-2\pi i s u w} du = s \int_{\R}
    f(u) e^{-2\pi i u (s w)} du = s \hat{f}(ws) . \]
\end{fact}

\begin{fact}[Transform inversion]
  \label{fact:fourier-inversion}
  For any $f \in L^{1}(\R)$,
  \[ f(x) = \int_{\R} \hat{f}(w) e^{2\pi i x w} dw . \]
\end{fact}
In words, this says that~$f(x)$ is a linear combination of
``character'' functions of the form $e^{2\pi i x w}$ for $w \in \R$,
each weighted by the Fourier coefficient $\hat{f}(w)$. Each component
$e^{2\pi i x w}$ is periodic with period $1/w$ (for $w \neq 0$), hence
frequency~$w$. Note that \cref{fact:fourier-inversion} implies that
$\hat{\hat{f}}(x) = f(-x)$.

\section{Fourier Series for Periodic Functions}
\label{sec:fseries}

Now consider a function~$g \colon \R \to \C$ (not necessarily
in~$L^{1}(\R)$) that is periodic with unit period, i.e.,
$g(x) = g(x + z)$ for any $z \in \Z$. We call such a function
\emph{$\Z$-periodic}. Equivalently, we can work with the function
$g \colon (\R / \Z) \to \C$, since the original~$g$ is constant over
any fixed coset $c + \Z$. Such an~$g$ can be decomposed as a linear
combination of character functions with \emph{integer} frequencies.

\begin{definition}[Fourier series]
  For $g \colon (\R / \Z) \to \C$, its Fourier series
  $\hat{g} \colon \Z \to \C$ is
  \[ \hat{g}(w) = \int_{\R/\Z} g(x+\Z) e^{-2\pi i x w} dx . \]
\end{definition}
Equivalently, the integral can be taken over any fundamental region
of~$\Z$; often it is convenient to use $[0,1)$.

\Cref{fact:linear,fact:shift,fact:scale} also apply to the Fourier
series, by essentially the same proofs. But there is a slightly
different inversion formula.

\begin{fact}[Series inversion]
  \label{fact:sinversion}
  \[ g(x+\Z) = \sum_{w \in \Z} \hat{g}(w) e^{2\pi i x w} . \]
\end{fact}

\begin{example}
  The Fourier series of $g(x) = e^{2\pi i x k}$, where~$k \in \Z$
  (so~$g$ is $\Z$-periodic), can be derived in two different ways.
  First, we calculate the Fourier series directly: for any $w \in \Z$,
  \begin{align*}
    \hat{g}(w) &= \int_{[0,1)} e^{2\pi i x k} e^{-2\pi i x w} dx \\
               &= \int_{[0,1)} e^{2\pi i x (k - w)} dx .
  \end{align*}
  When $w \neq k$, $e^{2\pi i x (k - w)} dx$ completes a nonzero
  integer number of revolutions around the unit circle (as~$x$ goes
  from~$0$ to~$1$), and thus the above integral is~$0$. When $k=w$,
  the integral is simply $\int_{[0,1)} e^{0} dx = 1$. Therefore
  $\hat{g}(w) = \delta_{k,w}$, where $\delta_{k,w}$ is the Kronecker
  delta function.

  Alternatively, we can match coefficients for each character function
  $e^{2\pi i x w}$ in the series inversion formula. We observe that if
  $e^{2\pi i x k} = \sum_{w \in \Z} \hat{g}(w) e^{2\pi i x w}$, we must
  have $\hat{g}(w) = \delta_{k,w}$.
\end{example}

\begin{example}
  The Fourier series of $g(x) = \cos(2\pi x)$ can be obtained by
  recalling that
  $\cos(2 \pi x) = \tfrac12 e^{2\pi i x} + \tfrac12 e^{-2\pi i x}$. So
  by matching coefficients in the series inversion formula, we have
  that $\hat{g}(1) = \hat{g}(-1) = \tfrac12$, and $\hat{g}(w)=0$ for
  $w \notin \pmone$.
\end{example}

\paragraph{Periodization.}

Let $f \in L^{1}(\R)$, so it has a Fourier transform. For a countable
set~$S$, define the notation $f(S) = \sum_{x \in S} f(x)$. Now
``$\Z$-periodize''~$f$ by summing all its $\Z$-translates, i.e.,
define $g \colon (\R/\Z) \to \C$ as
\begin{equation}
  \label{eq:periodize}
  g(x + \Z) := f(x + \Z) = \sum_{z \in \Z} f(x+z) .
\end{equation}

\begin{lemma}
  \label{lem:periodize}
  The Fourier series of~$g$ is $\hat{g}(w) = \hat{f}(w)$.
\end{lemma}

Intuitively, this says that periodization by~$\Z$ ``zeroes out'' all
the non-integer frequencies, and preserves all the integer ones.

\begin{proof}
  For any $w \in \Z$, we have
  \begin{align*}
    \hat{g}(w)
    &= \int_{\R/\Z} g(x+\Z) e^{-2\pi i x w} dx \\
    &= \int_{[0,1)} \sum_{z \in \Z} f(x+z) e^{-2\pi i x w} dx \\
    &= \int_{[0,1)} \sum_{z \in \Z} f(x+z) e^{-2\pi i (x+z) w} dx
    & \text{($z,w$ are integers)} \\
    &= \int_{\R} f(u) e^{-2\pi i u w} du
    & \text{($u=x+z$ runs over~$\R$)} \\
    &= \hat{f}(w) .
  \end{align*}
\end{proof}

\paragraph{Poisson summation.}

We can use the above to get an interesting formula involving the sum
of a function evaluated at the integers.

\begin{theorem}[Poisson summation formula]
  \label{thm:psf}
  For any $f \in L^{1}(\R)$, we have $f(\Z) = \hat{f}(\Z)$.
\end{theorem}

\begin{proof}
  Define $g \colon (\R/\Z) \to \C$ to be the $\Z$-periodization
  of~$f$, as in \cref{eq:periodize}.
  By~\cref{fact:sinversion,lem:periodize}, it follows that
  \[ f(\Z) = g(0+\Z) = \sum_{w \in \Z} \hat{g}(w) e^{2 \pi i 0 w} =
    \sum_{w \in \Z} \hat{g}(w) = \sum_{w \in \Z} \hat{f}(w) =
    \hat{f}(\Z) . \]
\end{proof}

More generally, to sum a function over a scaling of the integers
$s^{-1} \Z$ (i.e., a general one-dimensional lattice) for some real
$s > 0$, we can use \cref{fact:scale} to rescale: defining
$h(x) = f(x/s)$, by \cref{thm:psf} (Poisson summation) and
\cref{fact:scale}, we have
\begin{equation}
  \label{eq:gen-psf}
  f(s^{-1} \Z) = h(\Z) = \hat{h}(\Z) = s \hat{f}(s \Z) .
\end{equation}

\begin{example}
  Define $f(x) = e^{-\pi x^2}$, and
  $f_s(x) := f(x/s) = e^{-\pi (x/s)^2}$. We approximate
  $f_s(\Z)$ for somewhat large~$s$:
  \begin{align*}
    f_s(\Z) = f(s^{-1} \Z)
    &= s \hat{f}(s\Z)
    & \text{(\cref{eq:gen-psf})} \\
    &= s f(s\Z)
    & \text{($\hat{f} = f$ for the Gaussian)} \\
    &\approx s f(0)
    & \text{($f(sz) = e^{-\pi (sz)^{2}} \approx 0$ for $z \in \Z
      \setminus \set{0}$)} \\
    &= s.
  \end{align*}
\end{example}

% \bibliography{common/lattices,common/crypto}
% \bibliographystyle{common/alphaabbrvprelim}

\end{document}

%%% Local Variables: 
%%% mode: latex
%%% TeX-master: t
%%% End: 
