\documentclass[11pt]{article}

\usepackage{fullpage}
\usepackage{newtxtext}
\usepackage{microtype}
\usepackage{amsmath,amsfonts,amssymb}
\usepackage[amsmath,amsthm,thmmarks,hyperref]{ntheorem}
\usepackage[colorlinks]{hyperref}
\usepackage{fancyhdr}
\usepackage{mathtools}
\usepackage{enumitem}
\usepackage[svgnames]{xcolor}
\usepackage{tikz}
\usepackage{import}

% load after hyperref, algpseudocode to get proper behavior
\usepackage[capitalize,nameinlink,noabbrev]{cleveref}

\import{common/}{head.tex}
\import{common/slides/}{tikzhead.tex}
\definecolor{structure.fg}{rgb}{0.2,0.2,0.7} % from favorite beamer theme

% VARIABLES

\newcommand{\lecturenum}{1}
\newcommand{\lecturetopic}{Mathematical Background}
\newcommand{\scribename}{Sara Krehbiel}

% END OF VARIABLES

%%% COMMANDS FOR LECTURES/HOMEWORKS

\newcommand{\lecheader}{%
  \chead{\large \textbf{Lecture \lecturenum\\\lecturetopic}}

  \lhead{\small
    \textbf{\href{https://github.com/cpeikert/LatticesInCryptography}{Lattices
        in Cryptography}\\University of Michigan, Fall 2022}}

  \rhead{\small \textbf{Instructor:
      \href{http://www.eecs.umich.edu/~cpeikert/}{Chris Peikert}\\Scribe:
      \scribename}}

  \setlength{\headheight}{26pt}
  \setlength{\headsep}{16pt}
}

\newcommand{\hwheader}{%
  \chead{\Large \textbf{Homework \hwnum}}

  \lhead{\small
    \textbf{\href{https://github.com/cpeikert/LatticesInCryptography}{Lattices
        in Cryptography}\\University of Michigan, Fall 2022}}

  \rhead{\small \textbf{Instructor:
      \href{http://www.eecs.umich.edu/~cpeikert/}{Chris
        Peikert}\\Student: \studentname}}

  \setlength{\headheight}{26pt}
  \setlength{\headsep}{16pt}
  
  \headrule
}

\lecheader                      % execute lecture commands

\pagestyle{plain}               % default: no special header

\begin{document}

\thispagestyle{fancy}           % first page should have special header

% LECTURE MATERIAL STARTS HERE

\section{A Brief History of Lattices in Cryptography}
\label{sec:brief-hist}

Lattices have been used in mathematics going back at least to the 18th
century. However, computational aspects of lattices were not
investigated much until the early 1980s, when they were successfully
employed for breaking several proposed cryptosystems (among many other
applications). In the mid-1990s, lattices were first used in the
\emph{design} of cryptographic schemes, and the last roughly 15 years
has seen an explosion of new systems and applications. Here is a
timeline of some of the more significant developments in the use of
lattices in cryptography.

\begin{itemize}[itemsep=0pt]
\item 18th century through early 20th century: mathematicians such as
  Gauss and Lagrange use lattices in number theory (e.g., to give
  proofs of the quadratic reciprocity and four-square theorems);
  Minkowski greatly advances the study of lattices in his ``geometry
  of numbers.''

\item Early 1980s: Lenstra, Lenstra and Lov{\'a}sz discover their
  famous ``LLL'' basis-reduction algorithm, whose applications include
  factoring integer polynomials and breaking several cryptosystems.

\item Mid-1990s: Ajtai shows a remarkable ``worst-case to average-case
  reduction'' for lattice problems, yielding a cryptographic one-way
  function based on \emph{worst-case} hardness conjectures; his
  follow-up work with Dwork gives a public-key encryption scheme
  supported by similar security theorems. However, due to their
  inefficiency and complexity, at the time these schemes are mainly of
  theoretical interest.

  Concurrently, Hoffstein, Pipher and Silverman introduce the NTRU
  public-key encryption scheme (and somewhat later, a related digital
  signature scheme), which is practically quite efficient, but lacks
  any theoretical evidence of security. After extensive cryptanalysis,
  some of the most efficient parameter sets are broken (and various
  iterations of the signature schemes are completely broken), though
  NTRU encryption appears to remain secure in an asymptotic sense.

\item Early 2000s: researchers such as Regev and Micciancio
  dramatically simplify and improve the early theoretical works,
  obtaining much stronger security theorems and greatly improved
  efficiency.

\item 2007--present: several researchers (e.g., Gentry, Brakerski,
  Vaikuntanathan, Lyubashevsky, your instructor) build a surprisingly
  rich toolbox of lattice-based cryptographic constructions, including
  powerful objects like trapdoor functions, signature schemes,
  identity- and attribute-based encryption, fully homomorphic
  encryption, and much more.

\item 2016--present: organizations like the US National Institute of
  Standard and Technology (NIST), Google, Cloudflare, etc.\ solicit,
  test, and ultimately select lattice-based cryptosystems for
  the standardization and deployment of `post-quantum' cryptography.
\end{itemize}
In this course we will cover a great deal of this history, especially
the more modern developments.

\section{Mathematical Background}
\label{sec:math-background}

We start with the definition of a lattice.

\begin{definition}
  \label{def:lattice}
  An $n$-dimensional \emph{lattice} $\lat$ is a discrete additive
  subgroup of $\R^{n}$.
\end{definition}

Let's unpack this definition.  First, a lattice $\lat$ is a
\emph{subgroup} of $\R^{n}$, under standard component-wise addition.
Recall that $\lat$ is a subgroup if it contains the identity element
$\veczero \in \R^{n}$ (the all-zeros vector), and if for any $\vecx,
\vecy \in \lat$, we have $-\vecx \in \lat$ and $\vecx + \vecy \in
\lat$.

Second, a lattice is \emph{discrete}: this means that every $\vecx \in
\lat$ has some ``neighborhood'' in which $\vecx$ is the only lattice
point.  Formally, for every $\vecx \in \lat$ there exists some
$\epsilon > 0$ such that $(\vecx + \epsilon \ball) \cap \lat =
\set{\vecx}$, where $(\vecx + \epsilon \ball)$ denotes the (open) ball
of radius $\epsilon$ centered at $\vecx$. Note that when $\lat$ is a
group, this condition is equivalent to one in which the quantifiers
are reversed, i.e., there exists a single $\epsilon > 0$ that works
for all $\vecx \in \lat$.  Moreover, this is equivalent to the
existence of some $\epsilon > 0$ that works for the origin $\veczero
\in \lat$.

\begin{example}
  \label{ex:lattices}
  Let us consider several examples of lattices and non-lattices.
  \begin{enumerate}[itemsep=0pt]
  \item The singleton set $\set{\veczero} \subset \R^{n}$ is a lattice
    (for any positive integer $n$).
  \item The integers $\Z \subset \R$ form a $1$-dimensional lattice,
    and the integer grid $\Z^{n} \subset \R^{n}$ is an $n$-dimensional
    lattice.  The set $\Z \times \set{0} = \set{(z,0) : z \in \Z}
    \subseteq \R^{2}$ is a two-dimensional lattice, though its rank is
    only one (see \cref{def:rank} below).
  \item For any lattice $\lat$, its scaling $c \lat = \set{c \vecx :
      \vecx \in \lat}$ by any real $c$ is also a lattice, e.g., the
    even integers~$2\Z$. More generally, any linear transformation
    applied to a lattice is also a lattice.
  \item The set $\set{\vecx \in \Z^{n} : \sum_{i=1}^{n} x_{i} \in
      2\Z}$ is a lattice; it is often called the ``checkerboard'' or
    ``chessboard'' lattice, especially in two dimensions.  (See
    \cref{fig:lattices}.)
  \item The rationals $\Q \subset \R$ do \emph{not} form a lattice,
    because although they form a subgroup, it is not discrete: there
    exist rational numbers that are arbitrarily close to zero.
  \item The odd integers $2\Z+1$ do \emph{not} form a lattice, because
    although they are discrete, they do not form a subgroup of $\R$.
    (However, they do comprise a \emph{coset} of the lattice $2\Z$;
    more on this point later.)
  \item The group $G=\Z + \sqrt{2}\Z$ is \emph{not} a lattice, because
    it is not discrete: since $\sqrt{2}$ admits arbitrarily good
    rational approximations $a/b$, there are values $a-b\sqrt{2} \in
    G$ that are arbitrarily close to zero.
  \end{enumerate}
\end{example}

\begin{figure}[h!]
  \centering

  \begin{tikzpicture}[baseline]
    \draw[Gray] (-3.5,0) -- (3.5,0);
    \foreach \x in {-3, ..., 3} {
      \draw (\x,-.15) -- (\x,.15) node[below=7pt] {$\x$};
      \path[latt] (\x,0) circle (2pt);
    }
  \end{tikzpicture}
  \quad
  \begin{tikzpicture}[baseline]
    \useasboundingbox[clip] (-3.5,-1.3) rectangle (3.5,1.3);
    
    \draw[Gray] (-3.5,-1.3) grid (3.5,1.3); \draw node[above
    left,offlabel] {$\mathcal{O}$};

    \foreach \x in {-3,-1,..., 3} {
      \path[latt] (\x,1) circle (2pt);
      \path[latt] (\x+1,0) circle (2pt);
      \path[latt] (\x,-1) circle (2pt);
    }
  \end{tikzpicture}

  \caption{The integer and checkerboard lattices.}
  \label{fig:lattices}
\end{figure}

\begin{definition}
  \label{def:rank}
  The \emph{rank} $k$ of a lattice $\lat \subset \R^{n}$ is the
  dimension of its linear span, i.e., $k=\dim(\spn(\lat))$.  When
  $k=n$, the lattice is said to be \emph{full rank}.
\end{definition}

In this course, all lattices are implicitly full rank unless stated
otherwise. This assumption is usually (but not always!) without loss
of generality, since we can restrict our attention to the subspace
$\spn(\lat)$, which can be mapped bijectively to~$\R^{k}$ under a
linear transformation that preserves inner products (and thus
Euclidean norms, volumes, etc.).

Of particular interest are \emph{rational} and \emph{integer}
lattices, which are subsets of $\Q^n$ and $\Z^n$, respectively.  (Note
that not all lattices are rational, e.g., $\sqrt{2} \Z$.)  Up to
scale, rational and integer lattices are equivalent: the discreteness
of a rational lattice $\lat$ guarantees a finite ``common
denominator''~$d$ such that the scaled lattice~$d \lat$ is integral.
Rationality is important for computational purposes, because we need
to be able to represent the lattice in an algorithm. However, most of
the mathematical results in this course will apply to arbitrary
(potentially non-rational) lattices, so we will explicitly identify
when we are restricting to integral lattices.

\subsection{Lattice Bases}
\label{sec:bases}

Except for the degenerate case $\set{\veczero}$, a lattice is always
an infinite set.  However, a basic fact (which we will not prove) is
that a lattice can always be finitely represented by a \emph{basis}.

\begin{definition}
  \label{def:basis}
  A \emph{basis} $\matB=\set{\vecb_1,\dots, \vecb_n} \subset \R^n$ of
  a lattice $\lat$ is a set of linearly independent vectors whose
  integer linear combinations generate the lattice:
  \begin{equation}
    \label{eq:lattice-basis}
    \lat = \lat(\matB) := \set[\bigg]{\sum_{i=1}^n z_i \vecb_i : z_i\in\Z}.
  \end{equation}
  Equivalently, if we interpret $\matB \in \R^{n \times n}$ as a
  nonsingular matrix whose (ordered) columns are $\vecb_1, \ldots,
  \vecb_n$, then $\lat = \matB \cdot \Z^n = \set{\matB \vecz : \vecz
    \in \Z^{n}}$.
\end{definition}

Another way of reading the above definition is that any lattice can be
obtained by applying some nonsingular linear transformation to the
integer lattice $\Z^{n}$.

While a lattice always has a basis, such a basis is not unique.
Essentially, the existence of multiple lattice bases is what makes
lattice-based cryptography possible!  (More on this later.)  Recall
that an integer matrix is said to be \emph{unimodular} if it
determinant is $\pm 1$, i.e., if $\matU^{-1}$ exists and is also an
integer matrix.

\begin{lemma}
  \label{lem:basis-unimod}
  Bases $\matB_1,\matB_2$ generate the same lattice $\lat$ if and only
  if there exists a unimodular $\matU \in \Z^{n \times n}$ such that
  $\matB_1 = \matB_2 \matU$.
\end{lemma}

\begin{proof}
  Suppose $\lat(\matB_1)=\lat(\matB_2)$, or equivalently, $\matB_1
  \cdot \Z^n=\matB_2 \cdot \Z^n$.  Then each column of $\matB_{1}$ is
  an integer combination of the columns of $\matB_{2}$, and
  vice-versa.  That is, there exist $\matU,\matV \in \Z^{n\times n}$
  such that $\matB_1=\matB_2 \matU$ and $\matB_2=\matB_1 \matV =
  \matB_{2} \matU \matV$.  Because $\matB_{2}$ is invertible, we have
  $\matU \matV=\matI$ and therefore $\det(\matU) \det(\matV) = 1$, so
  $\det(\matU) = \det(\matV)=\pm 1$ because both matrices are integral.

  For the other direction, suppose $\matB_1=\matB_2 \matU$ for some
  unimodular $\matU$. Then we have \[ \lat(\matB_1)=\matB_1 \cdot \Z^n
  = \matB_2 \cdot (\matU \cdot \Z^n) = \matB_2 \cdot \Z^n =
  \lat(\matB_{2}), \] where the equality $\matU \cdot \Z^{n} = \Z^{n}$
  follows from the fact that $\matU \vecx = \vecy \iff \vecx =
  \matU^{-1} \vecy$, and that $\matU, \matU^{-1}$ are both integral.
\end{proof}

\begin{corollary}
  \label{cor:bases-Zn}
  The bases of $\Z^{n}$ are exactly the unimodular matrices $\matU \in
  \Z^{n \times n}$.
\end{corollary}

\begin{corollary}
  \label{cor:check-same-lattice}
  We can efficiently test whether two given matrices
  $\matB_{1}, \matB_{2} \in \Z^{n \times n}$ generate the same
  lattice, by checking whether $\matB_1^{-1} \cdot \matB_2$ is
  unimodular.
\end{corollary}

\subsection{Fundamental Regions}
\label{sec:fundamental-regions}

Since a lattice is an infinite periodic ``grid'' in $\R^{n}$, it is
often useful to consider a corresponding periodic ``tiling'' of
$\R^{n}$ by copies of some body. For example, consider the typical
tiling of bricks in a wall (which corresponds to the checkerboard
lattice), or a tiling of the plane by regular hexagons (with lattice
points at their centers). The notion of a fundamental region
formalizes this concept.

\begin{definition}
  \label{def:fund-region}
  A set $\calF \subseteq \R^{n}$ is a \emph{fundamental region} of a
  lattice $\lat$ if its translates
  $\vecx+\calF := \set{\vecx + \vecy : \vecy \in \calF}$, taken over
  all $\vecx \in \lat$, form a partition of $\R^n$.
\end{definition}

For example, the half-open unit intervals $[0,1)$ and $\hohalf$ are
fundamental regions of the integer lattice~$\Z$: any $x \in \R$ is in
the unique translate $\floor{x} + [0,1)$ or $\round{x} + \hohalf$,
respectively (where $\round{x} := \floor{x + \tfrac12}$ denotes
rounding to the nearest integer, with ties broken upward).  Similarly,
the half-open cubes $[0,1)^{n}$ and $\hohalf^{n}$ are fundamental
regions of $\Z^{n}$.  Note that in general, a fundamental region need
not be convex or even consist of a single connected component (though
all the fundamental regions we consider in this course will satisfy
both of those properties).

A lattice basis naturally yields a fundamental region:

\begin{definition}
  \label{def:fund-piped}
  The \emph{fundamental parallelepiped} of a lattice basis~$\matB$ is
  defined as
  \begin{equation}
    \label{eq:fund-piped}
    \piped(\matB) := \matB \cdot \hohalf^n =
    \set[\bigg]{\sum_{i=1}^{n} c_i \vecb_i : c_i \in \hohalf}.
  \end{equation}
\end{definition}
    
Alternatively, the fundamental parallelepiped is sometimes defined as
$\piped(\matB) = \matB \cdot [0,1)^n$.  However, in this course we
will typically use the formulation from
\cref{eq:fund-piped}, since it has the convenient property
of being essentially symmetric about the origin (ignoring its
boundary).

\begin{lemma}
  \label{lem:fund-piped-region}
  Let $\calF$ be a fundamental region of $\Z^{n}$ and $\matB$ be a
  lattice basis.  Then $\matB \cdot \calF = \set{\matB \vecx : \vecx
    \in \calF}$ is a fundamental region of $\lat = \lat(\matB)$.  In
  particular, $\piped(\matB)$ is a fundamental region of $\lat$.
\end{lemma}

\begin{proof}
  We need to show that each $\vecx \in \R^n$ is in exactly one
  translate $\vecv + \matB \cdot \calF$, where
  $\vecv = \matB \vecz \in \lat$ for some $\vecz \in \Z^{n}$. Because
  $\matB$ is nonsingular, we have
  $\vecx \in \matB \vecz + \matB \cdot \calF$ if and only if
  $\matB^{-1} \vecx \in \vecz + \calF$. Since
  $\matB^{-1} \vecx \in \R^{n}$ and $\calF$ is a fundamental region of
  $\Z^{n}$, there is exactly one $\vecz \in \Z^{n}$ for which the
  latter inclusion holds, which proves the claim.
\end{proof}

\begin{figure}[h!]
  \centering
  \begin{tikzpicture}
    \coordinate (sw) at (-4,-1.7);
    \coordinate (ne) at (4,2.2);
    % clip to size we want
    \useasboundingbox[clip] ([shift={(-2pt,-2pt)}] sw) rectangle ([shift={(2pt,2pt)}] ne);

    \begin{scope}[Ztrans]
      \fill[openball] (.5,-.5) -- (.5,.5) -- (-.5,.5);
      \draw[ball] (-.5,.5) -- (-.5,-.5) -- (.5,-.5);

      \coordinate (off0) at (0,0);
      \coordinate (off1) at (1,0);
      \coordinate (off2) at (1,-1);
      \coordinate (off3) at (-1,0);
      \coordinate (off4) at (-2,1);
      \coordinate (off5) at (-1,1);
      \coordinate (off6) at (0,1);

      % tiling with fundamental parallelepipeds
      \foreach \off in {off2,off3,off0,off1,off4,off5,off6} {
        \fill[openball] ($(\off)+(.5,-.5)$) -- ($(\off)+(.5,.5)$) -- ($(\off)+(-.5,.5)$);
        \draw[ball] ($(\off)+(-.5,.5)$) -- ($(\off)+(-.5,-.5)$) -- ($(\off)+(.5,-.5)$);
      }
    \end{scope}

    \begin{scope}[Ztrans]
      \path[latt] \latt{2pt};

      % some basis vectors
      \draw[->] (0,0) -- (1,0) node[above right,offlabel]
      {$\vecb_1$};
      \draw[->] (0,0) -- (0,1) node[above right,offlabel]
      {$\vecb_2$};
    \end{scope}
  \end{tikzpicture}

  \caption{A lattice basis with a (partial) tiling by its fundamental
    parallelepiped.}
  \label{fig:bases-fund-pipeds}
\end{figure}

Another useful fundamental region is given by the \emph{Voronoi cell}
of a lattice, which is the set of all points in~$\R^{n}$ that are
closer to the origin than to any other lattice point:
\begin{equation}
  \label{eq:voronoi}
  \mathcal{V}(\lat) := \set{\vecx \in \R^n : \length{\vecx} <
    \length{\vecx-\vecv}\ \forall\ \vecv \in \lat \setminus \set{\veczero} }.
\end{equation}
Actually, the open set $\mathcal{V}(\lat)$ itself is not quite a
fundamental region: its translates by lattice points are pairwise
disjoint and cover all of $\R^{n}$, \emph{except} for those points on
the boundaries of the translates (which form a set of measure zero).
The Voronoi cell can be made into a true fundamental region by making
it ``half-open'' in some appropriate sense, but this involves some
technical subtleties.

A basic fact is that all fundamental regions of a lattice have the
\emph{same} volume; this volume is an essential lattice invariant.

\begin{definition}
  \label{def:volume}
  The \emph{volume} (or \emph{determinant}) of a lattice $\lat$,
  denoted $\det(\lat)$, is defined as $\vol(\calF)$ where $\calF$ is
  any fundamental region of $\lat$.
\end{definition}

\begin{claim}
  \label{clm:vol-fund-piped}
  The volume of a lattice $\lat$ is $\det(\lat) = \abs{\det(\matB)}$,
  where $\matB$ is any basis of $\lat$.
\end{claim}

\begin{proof}
  By \cref{lem:fund-piped-region}, $\piped(\matB)$ is a
  fundamental region of $\lat$, and by basic linear algebra,
  $\vol(\piped(\matB)) = \abs{\det(\matB)}$.
\end{proof}

Note that $\abs{\det(\matB)}$ is invariant under choice of lattice
basis $\matB$, because by \cref{lem:basis-unimod}, any other
basis is of the form $\matB' = \matB \matU$ for some unimodular
$\matU$, and $\abs{\det(\matB')} = \abs{\det(\matB)} \cdot
\abs{\det(\matU)} = \abs{\det(\matB)}$.

\subsection{Minimum Distance}
\label{sec:minimum-distance}

Because a lattice $\lat$ is discrete, it has a nonzero element
$\vecv \in \lat$ of minimum length under some specified norm, e.g.,
the Euclidean norm. This element is not unique, because also
$-\vecv \in \lat$.

\begin{definition}
  \label{def:min-dist}
  The \emph{minimum distance} of a lattice $\lat$ is defined as
  \begin{equation}
    \label{eq:min-dist}
    \lambda_1(\lat) := \min_{\vecv \in \lat \setminus \set{\veczero}}
    \length{\vecv}=\min_{\text{distinct } \vecx,\vecy\in\lat} \length{\vecx-\vecy}.
  \end{equation}
  (The second equality follows from the fact that $\vecx-\vecy$ is
  itself a nonzero lattice vector.)
\end{definition}

A central result of Minkowski relates the minimum distance of a
lattice to its determinant.  Recall that a \emph{centrally symmetric}
body $S$ is one for which $\vecx \in S \iff -\vecx \in S$, and a
\emph{convex} body $S$ is one for which $\vecx, \vecy \in S
\Rightarrow \alpha \vecx + (1-\alpha) \vecy \in S$ for any $\alpha \in
[0,1]$, i.e., the line segment connecting $\vecx$ and $\vecy$ is
entirely contained in $S$.

\begin{theorem}[Minkowski]
  \label{thm:minkowski-body}
  Any convex, centrally symmetric body $S$ of volume $\vol(S) >
  2^n \cdot \det(\lat)$ contains a nonzero lattice point.
\end{theorem}

\begin{proof}
  Let $S'=S/2$, so $\vol(S') > \det(\lat)$. (Recall that in~$\R^{n}$,
  volumes scale by the $n$th power of the scaling factor.) By a
  ``volumetric pigeonhole argument,'' we claim that there exist
  distinct $\vecx, \vecy \in S'$ such that $\vecx-\vecy \in \lat$. To
  see this, consider any fundamental region $\calF$ of $\lat$, and
  partition $S'$ into sets $S'_{\vecv} := S' \cap (\vecv + \calF)$ for
  each $\vecv \in \lat$. Then the translates
  $S'_{\vecv} - \vecv \subseteq \calF$ have total volume
  $\vol(S') > \vol(\calF)$, so they must overlap somewhere, i.e.,
  there exists some
  $\vecz \in (S'_{\vecu} - \vecu) \cap (S'_{\vecv} - \vecv)$ for
  distinct lattice points $\vecu,\vecv \in \lat$. It follows that
  $\vecx = \vecz + \vecu, \vecy = \vecz + \vecv \in S'$ are two
  distinct points in $S'$ whose difference
  $\vecx-\vecy = \vecu-\vecv \in \lat$ is a lattice point.

  Finally, we have $2\vecx, -2\vecy\in S$ by definition of $S'$ and
  central symmetry of~$S$, and their midpoint $(2\vecx-2\vecy)/2 =
  \vecx-\vecy\in S$ by convexity.
\end{proof}

\begin{corollary}[Minkowski's First Theorem]
  \label{cor:minkowski-first}
  For any lattice $\lat$, we have $\lambda_1(\lat) \leq \sqrt{n} \cdot
  \det(\lat)^{1/n}$.
\end{corollary}

Notice that this bound ``scales properly'': if we scale the lattice
(and hence its minimum distance) by a~$c$ factor, the determinant
scales by a~$c^{n}$ factor, hence the bound also scales by a~$c$
factor.

\begin{proof}
  Without loss of generality, assume that $\det(\lat) = 1$ by scaling
  the lattice by a $\det(\lat)^{-1/n}$ factor, which scales
  $\lambda_{1}$ by the same factor.  Now take $S = \sqrt{n} \cdot
  \bar{\ball}$, the (closed) Euclidean ball of radius $\sqrt{n}$.
  The cube $[-1,1]^{n}$, which has side length $2$ and therefore
  volume $2^{n}$, is strictly contained in $S$, and so $\vol(S) >
  2^n$.  Therefore, $S$ contains a nonzero lattice point by
  \cref{thm:minkowski-body}, and the bound on $\lambda_{1}$
  follows.
\end{proof}

Note that the bound we used on the volume of a ball was somewhat
loose. Using a more precise formula for the volume of an
$n$-dimensional ball, we can obtain a slightly tighter bound
$\lambda_{1}(\lat) \leq \sqrt{n/(2 \pi e)} \cdot \det(\lat)^{1/n}$,
which is better than the one given in \cref{cor:minkowski-first} by
only a constant factor.

Also notice that Minkowski's bound can be arbitrarily loose: for
example, consider the lattice $\lat \subset \R^{2}$ of unit
determinant generated by the basis vectors $(2^{100}, 0)$,
$(0,2^{-100})$.  Then $\lambda_{1}(\lat) = 2^{-100} \ll \sqrt{2}$.
However, it is known that in general, Minkowski's bound cannot be
improved beyond small constant factors: there exist infinite families
of $n$-dimensional, unit-determinant lattices $\lat$ for which
$\lambda_{1}(\lat) \geq C \sqrt{n}$, where $C \approx \sqrt{1/(\pi
  e)}$ is some universal positive constant.

\end{document}

%%% Local Variables: 
%%% mode: latex
%%% TeX-master: t
%%% End: 
